\section{Funciones Características}
\begin{Def}
  Sean $X$ una variable aleatoria y $f$ su función de densidad o de masa
  (dependiendo si $X$ es continua o discreta). Se define la
  \emph{función característica} de $X$ como
  \[\phi_X(t) := \text{E}[e^{itX}] = \text{E}[\cos(tX)] + i\text{E}[\sin(tX)]\]
\end{Def}

\begin{Teo}
  La función característica existe para todo $t\in\R$ para
  toda variable aleatoria
  tanto discreta como absolutamente continua ($|A|\leq|\N|\To P(X\in A)=0$).
\end{Teo}
\begin{Demo}
  Sean $X$ una variable aleatoria y $f$ su función de densidad o de masa.
  Entonces, para todo $x$ en el dominio de $X$, $f(x)\geq0$,  luego,
  para todo $t\in\R$ y $x$ en el dominio de $X$,
  $|\cos(tx)f(x)|\leq f(x)$ y $|\sin(tx)f(x)|\leq f(x)$.
  Por criterio de convergencia absoluta, la serie o integral que
  define la función característica, converge para todo $t\in\R$.
\end{Demo}

\begin{Teo}\label{Teo:prop_carac}
  Sea $X$ una variable aleatoria. Entonces
  \begin{enumerate}
    \item $\phi_X(0)=1$.
    \item Para todo $t\in\R$, $|\phi_X(t)|\leq 1$.
    \item Sea $k\in\Z^+$. Si $\text{E}[X^k]$ existe, entonces,
    $\text{E}[X^k]=\left.\odv*[k]{\phi_X(t)}{t}\right|_{t=0}$
  \end{enumerate}
\end{Teo}
Solo se demostrará el primer enunciado
\begin{Demo}
  Sea $f$ la función de densidad o masa de $X$. En las expresiones que
  definen $\phi_X$, evaluadas en $0$, se obtiene
  $\cos(0x)f(x)=f(x)$ y $\sin(0x)f(x)=0$.
  Se sigue que la serie o integral resultante tenga como
  integrando únicamente a $f$, luego $\phi_X(0)=1$.
\end{Demo}
Por último, se deja enunciado el siguiente teorema
\begin{Teo}\label{Teo:biy_carac}
  Sean $X,Y$ dos variables aleatorias independientes. Si
  para todo $t\in\R$,
  \[\phi_X(t)=\phi_Y(t)\]
  entonces, $X$ y $Y$ tienen la misma distribución
\end{Teo}