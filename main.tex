\documentclass{article}

\usepackage[arrows]{logicDG}
\usepackage{calcDG}
\usepackage{analysis}
\usepackage{nicefrac}
\usepackage[a]{esvect}
\usepackage{amsfonts}
\usepackage{upgreek}
\usepackage{mathrsfs}
\usepackage{letltxmacro}
\usepackage{amsthm}
\usepackage{thmtools}
\usepackage{etoolbox}
\usepackage[hidelinks]{hyperref}
\usepackage{lmodern}
\usepackage[T1]{fontenc}
\usepackage[spanish, es-noquoting, es-lcroman, es-noshorthands]{babel}
\usepackage{marvosym}
\usepackage{csquotes}
\usepackage[sorting=none]{biblatex}
\addbibresource{Referencias.bib}
\usepackage{fancyhdr}
\usepackage{graphicx}
\usepackage{setspace}
\usepackage{enumerate}
\usepackage{tikz}
\usepackage{ifthen}
\usetikzlibrary{positioning}
\usepackage{geometry}
\geometry{
  left=2cm,
  right=2cm,
  bottom=4cm,
  a4paper
}
\hypersetup{
  colorlinks=false,
  pdftitle={DemostracionesPRYE},
  pdfauthor={David Gómez y Laura Rincón}
}
\renewcommand{\labelenumi}{(\roman{enumi})}

\pagestyle{fancy}
\setlength{\headheight}{13pt}
\fancyhf{}
\fancyhead[R]{\textit{David G., Laura R.}}
\fancyhead[L]{\leftmark}
\fancyfoot[C]{\thepage}

\declaretheoremstyle[
  spaceabove=10pt,
  spacebelow=20pt,
  bodyfont={\normalfont},
  notefont={\normalfont},
  notebraces={(}{)},
  headpunct={:},
  headfont={\bfseries},
  ]{definition}
  
\declaretheoremstyle[
    headpunct={:},
    headfont={\bfseries},
    bodyfont={\normalfont\leftskip2.5em},
    headindent=-2.5em,
    qed={\qedsymbol}
]{Proof}

% Entornos de definiciones, teoremas y demás %
%%%%%%%%%%%%%%%%%%%%%%%%%%%%%%%%%%%%%%%%%%%%%%
\declaretheorem[style=definition,name=Definición]{Def}
\declaretheorem[style=definition,name=Teorema]{Teo}
\declaretheorem[style=Proof, numbered=no, name=Demostración]{Demo}
\declaretheorem[style=definition, numbered=no, name=Lema]{Lema}
% Portada %
\newcommand*{\titleTMB}{\begingroup
\def\drop{0.1\textheight}
\centering
\vspace*{\baselineskip}
{\large\scshape David Gómez, Laura Rincón}\\[\baselineskip]
\rule{\textwidth}{1.6pt}\\[0pt]
\vspace*{-\baselineskip}
\vspace*{2pt}
\rule{\textwidth}{0.4pt}\\[\baselineskip]
{\LARGE DEMOSTRACIONES DE PRYE}\\[\baselineskip]
\rule{\textwidth}{0.4pt}\\[0pt]
\vspace*{-\baselineskip}\vspace{3.2pt}
\rule{\textwidth}{1.6pt}\\[\baselineskip]
\vfill
{\large\scshape Matemáticas}\\[\baselineskip]
{\small\scshape 2024}\par
\vspace*{\drop}
\endgroup}

% Comandos y demás %
%%%%%%%%%%%%%%%%%%%%%%%%%%
\everymath{\displaystyle}
\RenewDocumentCommand{\dfrac}{mm}{\frac{\displaystyle#1}{\displaystyle#2}}
\renewcommand{\epsilon}{\upvarepsilon}
\renewcommand{\iff}{\Leftrightarrow}
\doublespacing
\hyphenpenalty=1000

\makeatletter
\let\oldr@@t\r@@t
\def\r@@t#1#2{%
\setbox0=\hbox{$\oldr@@t#1{#2\,}$}\dimen0=\ht0
\advance\dimen0-0.2\ht0
\setbox2=\hbox{\vrule height\ht0 depth -\dimen0}%
{\box0\lower0.4pt\box2}}
\LetLtxMacro{\oldsqrt}{\sqrt}
\renewcommand*{\sqrt}[2][\ ]{\oldsqrt[#1]{#2}}
\makeatother

\begin{document}
\begin{titlepage}
  \titleTMB
\end{titlepage}
\tableofcontents
\clearpage

\section{Introducción}

La intención de este proyecto, como indíca el título, fue recompilar
todos los enunciados presentados en el curso de Probabilidad y Estadística
y demostrarlos. Este, si bien fue un proyecto para nosotros, 
es un trabajo recomendado para cualquiera que esté cursando dicha
asignatura, más aún si es del programa de matemáticas. Realizar
este proyecto permite entender de mejor manera los temas vistos y,
como opinión personal, hace más entretenida la asignatura.

Agradecemos al profesor Yesid Esteban Clavijo Penagos por darnos
este proyecto y por el esfuerzo de preparar actividades exclusivas
para los estudiantes del programa de matemáticas.

\begin{center}
  \LARGE{\Coffeecup}
\end{center}
\clearpage
\section{Probabilidad}
Un primer resultado en el curso es a cerca de la probabilidad de una unión finita
de eventos de un mismo espacio muestral. Para esto, se puede hacer uso del 
resultado para la unión de dos eventos, mismo que se puede aplicar
a una unión finita para generar una función recursiva.
\begin{align*}
  P(A_1 \cup A_2) &= P(A_1) + P(A_2) - P(A_1 \cap A_2)\\[10pt]
  P\left(\bigcup_{i=1}^{n+1} A_i\right)
  &= P(A_{n+1}) + P\left(\bigcup_{i=1}^n A_i\right) -
    P\left(\bigcup_{i=1}^n (A_{n+1}\cap A_i)\right)
\end{align*}
Para obtener una expresión más explícita, se desarrolla la recursión
hasta $n=4$. Claramente se omiten los resultados para $n=1$ y $n=2$.

\begin{longderivation}
    & {P(A_1 \cup A_2\cup A_3)}\\
  =\\
    & {P(A_3) + P(A_1 \cup A_2) - P((A_3\cap A_1) \cup (A_3\cap A_2))}\\
  =\\
    &P(A_3) + P(A_2) + P(A_1) - P(A_1\cap A_2)\\
    &-[P(A_1\cap A_3) + P(A_2 \cap A_3) - P(A_1\cap A_2\cap A_3)]\\
  =\\
    &P(A_1) + P(A_2) + P(A_3) - P(A_1\cap A_2) - P(A_1\cap A_3)\\
    &- P(A_2\cap A_3) + P(A_1\cap A_2\cap A_3)\\
  =\\
    & {
      \sum_{i=1}^3P(A_i) - \sum_{i=1}^2\sum_{j=i+1}^3P(A_i\cap A_j)
      + P\left(\bigcap_{i=1}^3 A_i\right)
    }
\end{longderivation}

Análogamente para cuatro eventos, usando lo obtenido

\begin{longderivation}
    & {P(A_1\cup A_2\cup A_3\cup A_4)}\\
  =\\
    & {P(A_4) + P(A_1\cup A_2\cup A_3) - P\left(\bigcup_{i=1}^3(A_4\cap A_i)\right)}\\
  =\\
    & P(A_4) + \sum_{i=1}^3P(A_i) - \sum_{i=1}^2\sum_{j=i+1}^3P(A_i\cap A_j)
    + P\left(\bigcap_{i=1}^3 A_i\right)\\
    &-\left[\sum_{i=1}^3P(A_4\cap A_i) - \sum_{i=1}^2\sum_{j=i+1}^3P(A_4\cap A_i\cap A_j)
    + P\left(\bigcap_{i=1}^4 A_i\right)\right]\\
  =\\
    & {
      \sum_{i=1}^4P(A_i) - \sum_{i=1}^3\sum_{i=1}^4P(A_i\cap A_j) +
      \sum_{i=1}^2\sum_{j=i+1}^3\sum_{k=j+1}^4P(A_1\cap A_j\cap A_k) -
      P\left(\bigcap_{i=1}^4 A_i\right)
    }
\end{longderivation}

Por último, recordar que
\[\sum_{i=a}^b\sum_{j=i+1}^{b+1} f(i,j) = \sum_{\mathclap{a\leq i < j \leq b}}f(i,j)\]
\begin{Teo}
  Sean $A_1,A_2,\dots,A_n$ eventos de un espacio muestral. Entonces,
  \[
    P\left(\bigcup_{i=1}^nA_i\right) = \smashoperator[r]{\sum_{1\leq i\leq n}}P(A_i)
    -\smashoperator[lr]{\sum_{1\leq i<j\leq n}}P(A_i\cap A_j) +
    \smashoperator[lr]{\sum_{1\leq i<j<k\leq n}}P(A_i\cap A_j\cap A_k)
    -\dots+(-1)^nP\left(\bigcap_{i=1}^n A_i\right)
  \]
  De otra forma, la probabilidad de una unión finita es la suma de
  la probabilidad de cada evento menos las posibles intersecciones
  dos a dos, sumando las probabilidades tres a tres\dots
\end{Teo}
\begin{Demo}
  Siguiendo por inducción. Los caso base $n=1$ y $n=2$ caen en la
  definición recursiva y para $n=3$ fue el desarrollo anterior.
  Para el paso inductivo, supóngase que la propiedad se mantiene hasta
  un valor $n$.
\begin{longderivation}<1>
    & {P\left(\bigcup_{i=1}^{n+1} A_i\right)}\\
  =\\
    & {
      P(A_{n+1}) + P\left(\bigcup_{i=1}^n A_i\right) -
      P\left(\bigcup_{i=1}^n (A_{n+1}\cap A_i)\right)
    }\\
  =\\
    &P(A_{n+1}) + P\left(\bigcup_{i=1}^n A_i\right)\\
    &-\left[
      \smashoperator[r]{\sum_{1\leq i\leq n}}(A_{n+1}\cap A_i)
      -\sum_{\mathclap{1\leq i<j\leq n}}(A_{n+1}\cap A_i\cap A_j) +
      \dots+(-1)^nP\left(\bigcap_{i=1}^{n+1}A_i\right)
    \right]\\
  =\\
    & 
        \smashoperator[r]{\sum_{1\leq i\leq n+1}}P(A_i)
        -\sum_{\mathclap{1\leq i < j\leq n+1}}P(A_i\cap A_j)\\
    &   +\\
    &   \smashoperator[r]{\sum_{1\leq i<j<k\leq n+1}}P(A_i\cap A_j\cap A_k)
        -\dots+(-1)^{n+1}P\left(\bigcap_{i=1}^{n+1} A_i\right)
\end{longderivation}
\end{Demo}

\begin{Def}
  Sean $A$ y $B$ eventos de un espacio muestral. Se dice que $A$ y $B$
  son independientes, cuando $P(A\cap B)=P(A)\,P(B)$.
\end{Def}

\begin{Teo}
  Sean $A$ y $B$ eventos independientes. Entonces
  \begin{enumerate}
    \item $A^c$ y $B^c$ son independientes.
    \item $A^c$ y $B$ son independientes.
  \end{enumerate}
\end{Teo}
\begin{Demo}
  \begin{enumerate}
    Supóngase $A$ y $B$ eventos independientes de un espacio muestral.
    \item Partiendo de que $A^c \cap B^c = (A\cup B)^c$.
    
    \begin{longderivation}<0.9>
        & {P(A^c\cap B^c)}\\
      =\\
        & {1 - P(A\cup B)}\\
      =\\
        & {1 - [P(A) + P(B) - P(A\cap B)]}\\
      =\\
        & {1-P(A)+P(B)-P(A)P(B)}\\
      =\\
        & {(1-P(A))(1-P(B))}\\
      =\\
        & {P(A^c)\,P(B^c)}
    \end{longderivation}
    
    Así, los eventos $A^c$ y $B^c$ también son independientes.
    \item Partiendo de que $B=(B\cap A)\cup(B\cap A^c)$
    
    \begin{longderivation}<0.9>
        & {P(B)}\\
      =\\
        & {P((B\cap A)\cup(B\cap A^c))}\\
      =\\
        & {P(B\cap A) + P(B\cap A^c) - P(\varnothing)}\\
      =\\
        & {P(B)\,P(A) + P(B\cap A^c)}
    \end{longderivation}
    
    Tomando la primera y última igualdad
    
    \begin{longderivation}<0.9>
        & {P(B\cap A^c)}\\
      =\\
        & {P(B) - P(B)P(A)}\\
      =\\
        & {P(B)(1-P(A))}\\
      =\\
        & {P(B)\,P(A^c)}
    \end{longderivation}
    Así, los eventos $A^c$ y $B$ también son independientes.
  \end{enumerate}
\end{Demo}

\clearpage
\section{Propiedades de la Varianza y el Valor Esperado}
\subsection{Propiedades da valor esperado}
\begin{Teo}
    Sea $X$ una variable aleatoria real entonces:
    \begin{enumerate}
        \item Si $P(X \geq 0) = 1$ y E$[X]$ existe entonces E[$X$]$\geq 0$
        \item E[$\alpha$]$= \alpha$ para $\alpha$ constante
        \item Si existe $M \geq 0$ tal que P($|X| \leq M$)$=1$ entonces E[$X$] existe. 
        \item Si $\alpha$ y $\beta$ son constantes, y si $g$ y $h$ son funciones tales que 
              $g(X)$ y $h(X)$ son variables aleatorias cuyos valores esperados existen, 
              entonces E[$\alpha g(X) + \beta h(X)$]$= \alpha$E[$g(X)$]$+ \beta$E[$h(X)$]
        \item Si $g$ y $h$ son funciones tales que $g(X)$ y $h(X)$ son variables aleatorias
              cuyos valores esperados existen y $g(x)\leq h(x)$ para todo $x$, entonces 
              E$[g(X)$]$\leq$E[$h(X)$]
        \item Sean $Y$ una variable aleatoria independiente de $X$ y $g,h$ funciones
        tales que $g(X)$ y $g(Y)$ sean variables aleatorias cuyos valores esperados
        existen. Entonces, $text{E}[g(X)h(Y)]=\text{E}[g(X)]\text{E}[h(Y)]$.
    \end{enumerate}
\end{Teo}
\begin{Demo} Sean $f,u$ las funciones de distribución para $X$ y $Y$
    respectivamente. Para hacer la demostración, se hará con variables discretas 
    y variables continuas por aparte.

\begin{enumerate}
    \item Para variables discretas:
    Si $P[X \geq 0]=1$ entonces $P(X < x)=0$, luego:
    \[
        \text{E}[X]=\sum_{x=0}^{\infty}xP(X=x) \geq  0
    \]
    Debido a que $x\geq0$ y $P(X=x)\geq 0$ para todo x.
    Similarmente para para variables aleatorias continuas, se tiene
    que $P(X\leq0)=0$. Entonces, como $f(x)\geq0$ para todo $x$,
    $xf(x)\geq0$ para $x\geq0$, y por tanto, E$[X]\geq0$.
    \item Para $\alpha$ constante, $P[X=\alpha]=1$, y por tanto el valor
    esperado es $\alpha$
    \item Supóngase que existe $M\in\R^+$ tal que $P(-M\leq X\leq M)=1$.
    Entonces:
    
    Para X variable aleatoria discreta, se tiene:
    \[
        \text{E}[X]=\sum_{-M}^{M}xP(X=x)
    \]
    y al ser una suma finita, el resultado existe. 
    
    Para X variable aleatoria contínua:
        \begin{longderivation}
            &\text{E}[X]\\
            =\\
            &\int_{-M}^{M}xf(x)\diff{x}\\
            \leq\\
            &M\int_{-M}^{M}f(x)\diff{x}
        \end{longderivation}
    Dado que $f$ es una función de densidad,
    por la condición de la que se partió,
    \begin{longderivation}
        &M\int_{-M}^{M}f(x)\diff{x}\\
        =\\
        &M\int_{-\infty}^{\infty}f(x)\diff{x}\\
        =\\
        &M
    \end{longderivation}
    Así pues, E$[X]\leq M$, y por lo tanto existe.
    \item Para funciones discretas, y tomando la 
    suma sobre $\Z$ bajo el orden usual,
    \begin{longderivation}
        &\text{E}[\alpha g(x) + \beta h(x)]\\
        =\\
        &\sum_{x\in\Z}(\alpha g(x) + \beta h(x))P[X=x]\\
        =\\
        &\sum_{x\in\Z}\alpha g(x) P[X=x] + 
        \sum_{x\in\Z} \beta h(x)P[X=x]\\
        =\\
        &\alpha \sum_{x\in\Z}g(x)P[X=x] + 
        \beta \sum_{x\in\Z}h(x)P(X=x)\\
        =\\
        &\alpha\text{E}[g(x)]+\beta\text{E}[f(x)]
    \end{longderivation}
    Para las variables continuas, es un argumento similar
    \begin{longderivation}
        &\text{E}[\alpha g(x) + \beta f(x)]\\
        =\\
        &\int_{-\infty}^{\infty}(\alpha g(x) + \beta h(x)\diff{x})\\
        =\\
        &\int_{-\infty}^{\infty}\alpha g(x)f(x)\diff{x} + 
        \int_{-\infty}^{\infty} \beta h(x)f(x)\diff{x}\\
        =\\
        &\alpha \int_{-\infty}^{\infty}g(x)f(x)\diff{x} + 
        \beta \int_{-\infty}^{\infty}h(x)f(x)\diff{x}\\
        =\\
        &\alpha\text{E}[g(x)]+\beta\text{E}[h(x)]
    \end{longderivation}
    \item Para variables aleatorias discretas:
    \begin{longderivation}
        &\text{E}[g(x)]\\
        =\\
        &\sum_{x\in\Z}g(x)P(X=x)\\
        \leq\\
        &\sum_{x\in\Z}h(x)P(X=x)\\
        =\\
        &\text{E}[h(x)]
    \end{longderivation}
    De manera similar, con las variables aleatorias continuas
    \begin{longderivation}
        &\text{E}[g(x)]\\
        =\\
        &\int_{-\infty}^{\infty}g(x)f(x)\diff{x}\\
        \leq\\
        &\int_{-\infty}^{\infty}h(x)f(x)\diff{x}\\
        =\\
        &\text{E}[h(x)]
    \end{longderivation}
\item tanto para el caso discreto como continuo, la expresión a sumas o integrar,
es $g(x)h(y)f(x)u(y)$, donde la expresión se suma o integra sobre el dominio
de $X$ y $Y$ simultaneamente. Por la naturaleza de las integrales y series, la
independencia de las expresiones permite separar la operacion en un producto,
el cual resulta ser E$[g(X)]$E$[h(Y)]$
Lo que termina la demostración.
\end{enumerate}
\end{Demo}
\subsection{Propiedades de varianza}
\begin{Teo}
    Sea $X$ una variable aleatoria cuyo valor esperado existe y 
    $\alpha$, $\beta \in \R$ constantes. Entonces:

    \begin{enumerate}
        \item Var$[X] \geq 0$ 
        \item Var$[\alpha]$ = 0
        \item Var$[\alpha X]$ = $\alpha^2$Var[X]
        \item Var$[X + \beta]$ = Var$[X]$
        \item Var$[X] = 0$ si y solo si $P(X=E(X))=1$ 
    \end{enumerate}
\end{Teo}
\begin{Demo}
    La demostración se hará en base a que Var$[X] =$ 
    E$[X^2] -$$\text{E}^2[X] =$E$[(X-E(X))^2]$
    \begin{enumerate}
        \item Por propiedad de valor esperado, como $(X-E(x))^2\geq 0$, 
        entonces E$[(X-E(X))^2]\geq 0$.  
        \item $\text{Var}[\alpha] = \text{E}[\alpha^2]-\text{E}^2[\alpha]
        =\alpha^2-\alpha^2=0$
        \item Se usa la misma propiedad del item anterior.
        
        \begin{longderivation}
        &\text{Var}[\alpha X]\\
        =\\
        &\text{E}[\alpha^2X^2] - \text{E}^2[\alpha X]\\
        =\\
        &\alpha^2\text{E}[X^2] - (\alpha\text{E}[X])^2\\
        =\\
        &\alpha^2(\text{E}[X^2]-\text{E}^2[X])\\
        =\\
        &\alpha^2\text{Var}[X]
        \end{longderivation}
        
        \item Por definición de varianza y propiedades del valor esperado
        \begin{longderivation}
            &\text{Var}[X+\beta]\\
            =\\
            &\text{E}[(X+\beta)^2]-\text{E}^2[x+\beta]\\
            =\\
            &\text{E}[X^2+2\beta X + \beta^2]-(\text{E}[X]+\text{E}[\beta])^2\\
            =\\
            &\text{Var}[X] + 2\beta\text{E}[X] + \beta^2
            -2\beta\text{E}[X]-\beta^2\\
            =\\
            &\text{Var}[X]
        \end{longderivation}
        \item $P(X=\text{E}[X]) =1$ es lo mismo que decir que 
        $X=\text{E}[X]$, esto es,$X$ es constante. Por el teorema del 
        valor esperado, E$[\text{E}[X]^2]=\text{E}^2[X]$, y por lo tanto
        $\text{Var}[X]=\text{E}^2[X]-\text{E}^2[X]=0$
    \end{enumerate}
\end{Demo}

\section{Funciones Características}
\begin{Def}
  Sean $X$ una variable aleatoria y $f$ su función de densidad o de masa
  (dependiendo si $X$ es continua o discreta). Se define la
  \emph{función característica} de $X$ como
  \[\phi_X(t) := \text{E}[e^{itX}] = \text{E}[\cos(tX)] + i\text{E}[\sin(tX)]\]
\end{Def}

\begin{Teo}
  La función característica existe para todo $t\in\R$ para
  toda variable aleatoria
  tanto discreta como absolutamente continua ($|A|\leq|\N|\To P(X\in A)=0$).
\end{Teo}
\begin{Demo}
  Sean $X$ una variable aleatoria y $f$ su función de densidad o de masa.
  Entonces, para todo $x$ en el dominio de $X$, $f(x)\geq0$,  luego,
  para todo $t\in\R$ y $x$ en el dominio de $X$,
  $|\cos(tx)f(x)|\leq f(x)$ y $|\sin(tx)f(x)|\leq f(x)$.
  Por criterio de convergencia absoluta, la serie o integral que
  define la función característica, converge para todo $t\in\R$.
\end{Demo}

\begin{Teo}\label{Teo:prop_carac}
  Sea $X$ una variable aleatoria. Entonces
  \begin{enumerate}
    \item $\phi_X(0)=1$.
    \item Para todo $t\in\R$, $|\phi_X(t)|\leq 1$.
    \item Sea $k\in\Z^+$. Si $\text{E}[X^k]$ existe, entonces,
    $\text{E}[X^k]=\left.\odv*[k]{\phi_X(t)}{t}\right|_{t=0}$
  \end{enumerate}
\end{Teo}
Solo se demostrará el primer enunciado
\begin{Demo}
  Sea $f$ la función de densidad o masa de $X$. En las expresiones que
  definen $\phi_X$, evaluadas en $0$, se obtiene
  $\cos(0x)f(x)=f(x)$ y $\sin(0x)f(x)=0$.
  Se sigue que la serie o integral resultante tenga como
  integrando únicamente a $f$, luego $\phi_X(0)=1$.
\end{Demo}
Por último, se deja enunciado el siguiente teorema
\begin{Teo}\label{Teo:biy_carac}
  Sean $X,Y$ dos variables aleatorias independientes. Si
  para todo $t\in\R$,
  \[\phi_X(t)=\phi_Y(t)\]
  entonces, $X$ y $Y$ tienen la misma distribución
\end{Teo}
\section{Distribuciones}

En esta sección se repasarán las definiciones de algunas distribuciones
de la probabilidad.

\subsection{Distribuciones Discretas}
Las distribuciones discretas son aquellas cuyas funciones de masa tienen
dominio en algún conjunto discreto. Esto se extenderá para esta
sección a los enteros, lo cual se asumirá a lo largo de
las definiciones y demostraciones a cerca de estas distribuciones.
De no ser especificado el valor de una función de masa en algún
subconjunto de $\Z$, se asumirá un como nulo.

\subsubsection{Binomial}
\label{dist:binom}
Supóngase que se realiza un experimento el cual tiene como posible
resultado $a$ o $b$ exclusivamente, y además, el resultado
de realizar nuevamente el experimento es independiente al
resultado anterior. Dado que $a$ y $b$ son los únicos resultados,
para un único experimento, se debe tener que
$P(a) = 1 - P(b)$. Sea $p=P(a)$. Supóngase que
este experimento es realizado $n$ veces. Se define una variable
aleatoria $X$ correspondiente a la cantidad de ocurrencias de $a$.
Entonces
\[P(X=x) = \binom{n}{x}p^x(1-p)^{n-x}\]
\begin{Def}
  Sea $X$ una variable aleatoria discreta. $X$ sigue una distribución
  binomial con parámetros $n$ y $p$ ($n\in\Z^+, p\in(0,1)$), denotada por $B(n,p)$,
  cuando su función de masa es
  \[f(x) = P(X=x) = B(n,p)(x) = \binom{n}{x}p^x(1-p)^{n-x}
  \qquad(0\leq x \leq n)\]
\end{Def}

\begin{Teo}
  Sea $X\sim B(n,p)$. Entonces
  \begin{enumerate}
    \item Para todo $x\in\Z$ con $0\leq x\leq n$, $0 \leq P(X=x) \leq 1$.
    \item $\sum_{x\in\Z}P(X=x) = 1$.
    \item $\text{E}[X] = np$.
    \item $\text{Var}[X]=np(1-p)$.
  \end{enumerate}
\end{Teo}
\begin{Demo}~
  \begin{enumerate}
    \item Sea $x\in\Z$ con $0\leq x\leq n$. Recordando que
    \[P(X=x) = \binom{n}{x}p^x(1-p)^{n-x}\]
    Dado todos los términos de la expresión son no negativos, se concluye que
    $P(X=x)\geq0$.
    Para la otra parte de la desigualdad, se usa el item (ii). En este se demuestra
    que la suma de todos los términos es $1$. Como todos los términos
    son no negativos, se concluye que cada uno debe ser menor o igual a $1$.
    \item~
    \begin{longderivation}
        & \sum_{x=0}^nP(X=x)\\
      =\\
        & \sum_{x=0}^n\binom{n}{x}p^x(1-p)^{n-x}\\
      =\\
        & (p + 1 - p)^n\\
      =\\
        & 1
    \end{longderivation}
    \item~
    \begin{longderivation}<0.8>
        & {\text{E}[X]}\\
      =\\
        & {\sum_{x=0}^nx\binom{n}{x}p^x(1-p)^{n-x}}\\
      =\\
        & {np\sum_{x=1}^n\frac{(n-1)!}{(n-k)!\,(x-1)!}p^x(1-p)^{n-x}}\\
      =\\
        & {np\sum_{x=0}^{n-1}\binom{n-1}{x}p^x(1-p)^{n-1-x}}\\
      =\\
        & {np\,(p+1-p)^{n-1}}\\
      =\\
        & {np}
    \end{longderivation}
    \item~
    \begin{longderivation}<0.8>
        & {\text{Var}[X]}\\
      =\\
        & {\text{E}[X^2] - \text{E}^2[X]}\\
      =\\
        & {\sum_{x=0}^nx^2\binom{n}{x}p^x(1-p)^{n-x} - (np)^2}\\
      =\\
        & {np\left[
          \sum_{x=1}^nx\binom{n-1}{x-1}p^{x-1}(1-p)^{n-x} - np
        \right]}\\
      =\\
        & {np\left[
          \sum_{x=0}^{n-1}(x+1)\binom{n-1}{x}p^x(1-p)^{n-1-x} - np
        \right]}\\
      =\\
        & {np\left[
          \sum_{x=0}^{n-1}x\binom{n-1}{x}p^x(1-p)^{n-1-x}
          +\sum_{x=0}^{n-1}\binom{n-1}{x}p^x(1-p)^{n-1-x}
          -np
        \right]}\\
      =\\
        & {np\left[
          p(n-1)\sum_{x=1}^{n-1}\binom{n-2}{x-1}p^{x-1}(1-p)^{n-1-x}+1-np
        \right]}\\
      =\\
        & {np\left[
          p(n-1)\sum_{x=0}^{n-2}\binom{n-2}{x}p^x(1-p)^{n-2-x}+1-np
        \right]}\\
      =\\
        & {np[p(n-1)+1-np]}\\
      =\\
        & {np(1-p)}
    \end{longderivation}
    Este argumento es válido siempre que $n\geq 2$. Si $n<2$, entonces $n=1$ y
    \begin{longderivation}<0.8>
        & \text{Var}[X]\\
      =\\
        & \text{E}[X^2] - \text{E}^2[X]\\
      =\\
        & \sum_{x=0}^1x^2\binom{1}{x}p^x(1-p)^{1-x} - p^2\\
      =\\
        & p - p^2\\
      =\\
        & p\,(1-p)
    \end{longderivation}
    Así, el resultado se mantiene para todo $n\in\Z^+$
  \end{enumerate}
\end{Demo}
\subsubsection{Hipergeométrica}
\label{dist:hip}
Supóngase que se tienen dos tipos de objetos, $a$ y $b$ en un total de
$N$ objetos exclusivamente de estos dos tipos. Sea $K$ el número de
objetos de tipo $a$ en el total de los $N$ objetos, es decir
hay $N-K$ objetos de tipo $b$. Supóngase que se toman ahora $n$
objetos del total ($N$). Se define una variable aleatoria $X$
correspondiente al número de objetos de tipo $a$ en los
$n$ objetos tomados. Entonces
\[P(X=x) = \dfrac{\binom{K}{x}\binom{N-K}{n-x}}{\binom{N}{n}}\]
\begin{Def}
  Sea $X$ una variable aleatoria discreta. $X$ sigue una distribución
  hipergeométrica con parámetros $N,K,n$ ($N,K,n\in\Z^+,K\leq N, n\leq N$),
  denotada por $Hg(N,K,n)$, cuando su función de masa es
  \[
    f(x)=P(X=x)=\dfrac{\binom{K}{x}\binom{N-K}{n-x}}{\binom{N}{n}}\qquad
    (\max\{0,n+K-N\} \leq x \leq \min\{K,n\})
  \]
\end{Def}

La razón de esta condición para $x$ está en que tenga sentido para lo
que se está representando. Por un lado, no tiene sentido pensar en la
probabilidad de tomar más objetos de tipo $a$ de los que hay en el total
de la muestra o tomar más objetos de tipo $a$ del total de estos.
De la misma forma, no tiene sentido tomar una cantidad negativa
de objetos tipo $a$, o tomar una cantidad de objetos tipo $a$
de forma que haya una cantidad negativa de objetos tipo $b$ para completar
los $n$ objetos o más objetos de tipo $b$ de los que hay en total.
De forma más concreta, se pueden ver las condiciones de $x$ dada la
expresión de la función de masa presentada.
\begin{longderivation}<0.8>
    & 0\leq x\leq K \quad\land\quad 0\leq n-x\leq N-K\\
  \iff\\
    & 0\leq x\leq K \quad\land\quad n+K-N\leq x\leq n\\
  \iff\\
    & \max\{0,n+K-N\}\leq x\leq\min\{K,n\}
\end{longderivation}
Sin embargo, tomando la convención de que $\binom{n}{k} = 0$ cuando $k>n$,
se puede tomar a $x$ entre $0$ y $n$.

Para demostrar la validez de esta función de masa, hace falta un resultado
sobre la combinatoria.
\begin{Lema}[Identidad de Vandermonde]
  Sean $m,n,k\in\Z$ no negativos. Entonces
  \[\binom{m+n}{k} = \sum_{r=0}^k\binom{m}{r}\binom{n}{k-r}\]
  Esta identidad tiene sentido tomando la convención mencionada anteriormente.
\end{Lema}
\begin{Demo}
  La demostración se hará por inducción sobre $n$, tomando $m,k$ como
  enteros no negativos arbitrarios. Si $k\geq n+m$, la propiedad
  resulta trivial. Supóngase que $0\leq k\leq n+m$.\\
  Caso base ($n=0$):
  \[\binom{m+0}{k} = \sum_{r=0}^k\binom{m}{r}\binom{0}{k-r}=\binom{m}{k}\]
  Paso inductivo:\\
  Nótese que para $k=0$, la identidad resulta en
  \[1 = \sum_{r=0}^0\binom{m}{r}\binom{n}{0-r}\]
  lo cual es cierto para todo $n,m\in\Z^+$. Supóngase que $k\geq1$,
  y que la identidad se mantiene para $m\geq0$. Entonces,
  \begin{longderivation}
      & \binom{m+n+1}{k+1}\\
    =\\
      & \binom{m+n}{k} + \binom{m+n}{k-1}\\
    =\\
      & \sum_{r=0}^k\binom{m}{r}\binom{n}{k-r} +
      \sum_{r=0}^{k-1}\binom{m}{r}\binom{n}{k-1-r}\\
    =\\
      & \binom{m}{k} + 
      \sum_{r=0}^{k-1}\binom{m}{r}\left(\binom{n}{k-r} + \binom{n}{k-1-r}\right)\\
    =\\
      & \binom{m}{n} + \sum_{k=0}^{k-1}\binom{m}{r}\binom{n}{k-r}\\
    =\\
      & \sum_{k=0}^k \binom{m}{k}\binom{n+1}{k-r}
  \end{longderivation}
\end{Demo}

\begin{Teo}
  Sea $X\sim Hg(N,K,n)$. Entonces,
  \begin{enumerate}
    \item para todo $x\in\Z$ con $0\leq x\leq n$, $0\leq P(X=x)\leq1$.
    \item $\sum_{x\in\Z}P(X=x) = 1$.
    \item $\text{E}[X]=\frac{nK}{N}$.
    \item $\text{Var}[X] = \frac{n\,K(N-K)(N-n)}{N^2(N-1)}$.
  \end{enumerate}
\end{Teo}
\begin{Demo}~
  \begin{enumerate}
    \item Sea $x\in\Z$ con $0\leq x\leq n$. Recordando que
    \[P(X=x) = \dfrac{\binom{K}{x}\binom{N-K}{n-x}}{\binom{N}{n}}\]
    dado que todos los términos de la expresión son no negativos, se
    concluye que $P(X=x)\geq0$.
    Para la otra parte de la desigualdad, recordando la identidad de Vandermonde,
    se tiene que
    \[\binom{N}{n} = \sum_{x=0}^n\binom{K}{x}\binom{N-K}{n-x}\]
    Dado que la suma presentada es de términos no negativos, entonces,
    para todo $x\in\Z$ con $0\leq x\leq n$, se tiene que
    \[\binom{N}{n} \geq \binom{K}{x}\binom{N-K}{n-x}\]
    \item ~
    \begin{longderivation}
        & \sum_{x=0}^nP(X=x)\\
      =\\
        & \sum_{x=0}^n\dfrac{\binom{K}{x}\binom{N-K}{n-x}}{\binom{N}{n}}\\
      =\\
        & \dfrac{1}{\binom{N}{n}}\sum_{x=0}^n\binom{K}{x}\binom{N-K}{n-x}\\
      =\\
        & \dfrac{\binom{N}{n}}{\binom{N}{n}}\\
      =\\
        & 1
    \end{longderivation}
    \item~
    \begin{longderivation}
        & \text{E}[X]\\
      =\\
        & \sum_{x=0}^nx\dfrac{\binom{K}{x}\binom{N-K}{n-x}}{\binom{N}{n}}\\
      =\\
        & \dfrac{1}{\binom{N}{n}}\sum_{x=0}^nx\binom{K}{x}\binom{N-K}{n-x}\\
      =\\
        & \dfrac{K}{\binom{N}{n}}\sum_{x=1}^n\binom{K-1}{x-1}\binom{N-K}{n-x}\\
      =\\
        & \dfrac{K}{\binom{N}{n}}\sum_{x=0}^{n-1}\binom{K-1}{x}\binom{N-K}{n-x-1}\\
      =\\
        & \dfrac{K}{\binom{N}{n}}\binom{N-1}{n-1}\\
      =\\
        & \frac{K\,n}{N}
    \end{longderivation}
    \item~
    \begin{longderivation}
        & \text{Var}[X]\\
      =\\
        & \text{E}[X^2] - \text{E}[X] + \text{E}[X] - \text{E}^2[X]\\
      =\\
        & \dfrac{1}{\binom{N}{n}}\sum_{x=0}^nx^2\binom{K}{x}\binom{N-K}{n-x}
        - \dfrac{1}{\binom{N}{n}}\sum_{x=0}^nx\binom{K}{x}\binom{N-K}{n-x}
        + \frac{n\,K}{N} - \frac{n^2\,K^2}{N^2}\\
      =\\
        & \dfrac{K}{\binom{N}{n}}\left[
          \sum_{x=0}^{n-1}x\binom{K-1}{x-1}\binom{N-K}{n-x-1} -
          \sum_{x=0}^{n-1}\binom{K-1}{x-1}\binom{N-K}{n-x-1}
        \right]
        + \frac{n\,K}{N} - \frac{n^2\,K^2}{N^2}\\
      =\\
        & \dfrac{K(K-1)}{\binom{N}{n}}
          \sum_{x=0}^{n-2}\binom{K-2}{x}\binom{N-K}{n-x-2}
        + \frac{n\,K}{N} - \frac{n^2\,K^2}{N^2}\\
      =\\
        & \dfrac{K(K-1)}{\binom{N}{n}}\binom{N-2}{n-2}
        + \frac{n\,K}{N} - \frac{n^2\,K^2}{N^2}\\
      =\\
        & \frac{K(K-1)n(n-1)}{N(N-1)}
        + \frac{n\,K}{N} - \frac{n^2\,K^2}{N^2}\\
      =\\
        & \frac{n\,K(N-K)(N-n)}{N^2(N-1)}
    \end{longderivation}
  \end{enumerate}
\end{Demo}
\subsubsection{Uniforme Discreta}
Algunas distribuciones surgen por la función de masa que las define, más que
por la similitud con un evento real, esto debido a resultados conocidos
sobre los enteros en este caso.

\begin{Def}
  Sea $X$ una variable aleatoria discreta. $X$ sigue una distribución uniforme
  discreta, de parámetros $n,m\in\Z$ ($n < m$), denotada por $U_d(n,m)$,
  cuando su función de masa es
  \[f(x) = P(X=x) = \frac{1}{m-n+1}\qquad (n\leq x\leq m)\]
\end{Def}
\begin{Teo}
  Sea $X\sim U(n,m)$. Entonces,
  \begin{enumerate}
    \item Para todo $x\in\Z$, $0\leq P(X=x)\leq1$.
    \item $\sum_{x\in\Z}P(X=x) = 1$.
    \item $\text{E}[X] = \frac{n+m}{2}$
    \item $\text{Var}[X] = \frac{(m-n+1)^2-1}{12}$
  \end{enumerate}
\end{Teo}
\begin{Demo}~
  \begin{enumerate}
    \item Se sigue inmediatamente de la definición.
    \item~
    \[\sum_{x=n}^m \frac{1}{n+m+1} = \frac{n+m-1}{n+m-1}=1\]
    \item~
    \begin{longderivation}
        & \text{E}[X]\\
      =\\
        & \sum_{x=n}^m\frac{x}{m-n+1}\\
      =\\
        & \frac{1}{m-n+1}\sum_{x=1}^{m-n+1}(x+n-1)\\
      =\\
        & \frac{1}{m-n+1}\left[\frac{(m-n+1)(m-n+2)}{2} + n(m-n+1) - (m-n+1)\right]\\
      =\\
        & \frac{m-n+2 + 2n - 2}{2}\\
      =\\
        & \frac{m+n}{2}
    \end{longderivation}
    \item Tomando $N=m-n+1$.
    \begin{longderivation}
        & \text{Var}[X]\\
      =\\
        & \text{E}[X^2] - \text{E}^2[X]\\
      =\\
        & \sum_{x=n}^m\frac{x^2}{m-n+1} - \left(\frac{n+m}{2}\right)^2\\
      =\\
        & \frac{1}{N}\sum_{x=1}^{m-n+1}(x+n-1)^2 - \left(\frac{n+m}{2}\right)^2\\
      =\\
        & \frac{1}{N}\left[\frac{N(N+1)(2N+1)}{6} + 2(n-1)\frac{N(N+1)}{2} + N(n-1)^2\right]
        - \left(\frac{n+m}{2}\right)^2\\
      =\\
        & (N+1)\frac{2N+1+6n - 6}{6} + (n-1)^2 - \left(\frac{n+m}{2}\right)^2\\
      =\\
        & (N+1)\frac{2N + 6n - 5}{6} + \left(n - 1 - \frac{n+m}{2}\right)
        \left(n-1+\frac{n+m}{2}\right)\\
      =\\
        & (N+1)\frac{2N + 6n - 5}{6}
        + \left(\frac{n-m-2}{2}\right)\left(\frac{3n+m-2}{2}\right)\\
      =\\
        & (N+1)\frac{2N + 6n - 5}{6} + \left(\frac{m-n+2}{2}\right)
        \left(\frac{2-3n-m}{2}\right)\\
      =\\
        & (N+1)\frac{2(2N + 6n - 5) + 3(2 - 3n + m)}{12}\\
      =\\
        & (N+1)\frac{4N - 3n - 3m -4}{12}\\
      =\\
        & (N+1)\frac{4m-4n+4-3n-3m-4}{12}\\
      =\\
        & \frac{N^2 - 1}{12}\\
      =\\
        & \frac{(m-n+1)^2-1}{12}
    \end{longderivation}
  \end{enumerate}
\end{Demo}
\subsubsection{Poisson}
Recordando que $e^x=\sum_{n=0}^\infty\frac{x^n}{n!}$, y que esta es una función
creciente cuyo valor es estrictamente positivo, se genera una función la cual
cumple la definición de función de masa, obteniendo la siguiente distribución.
\begin{Def}
  Sea $X$ una variable aleatoria discreta. $X$ sigue una distribución
  de Poisson con parámetro $\lambda\in\R^+$, denotada por $\text{Pois}(\lambda)$,
  cuando su función de masa es
  \[f(x) = P(X=x) = \frac{\lambda^xe^{-\lambda}}{x!} \qquad (x\in\N)\]
\end{Def}
\begin{Teo}
  Sea $X\sim\text{Pois}(\lambda)$. Entonces,
  \begin{enumerate}
    \item Para todo $x\in\Z$, $0\leq P(X=x)\leq1$.
    \item $\sum_{x\in\Z}P(X=x)=1$.
    \item $\text{E}[X] = \lambda$.
    \item $\text{Var}[X]=\lambda$.
  \end{enumerate}
\end{Teo}
\begin{Demo}~
  \begin{enumerate}
    \item Sea $x\in\N$. Entonces
    \[P(X=x) = \frac{\lambda^xe^{-\lambda}}{x!}\]
    Como los términos involucrados son no negativos, se concluye que $P(X=x)\geq0$.
    Para la otra parte de la desigualdad, dado que, para todo $\lambda\in\R$,
    \[e^{\lambda} = \sum_{n=0}^{\infty}\frac{\lambda^n}{n!}\]
    y la serie presentada es de términos no negativos para $\lambda\in\R^+$, se sigue
    que, para todo $x\in\N$,
    \[\frac{\lambda^x}{x!} \leq e^{\lambda}\]
    Así, para todo $x\in\N$,
    \[0\leq P(X=x)\leq1\]
    \item~
    \begin{longderivation}<1>
        & \sum_{x=0}^\infty\frac{\lambda^xe^{-\lambda}}{x!}\\
      =\\
        & e^{-\lambda}\sum_{x=0}^\infty\frac{\lambda^x}{x!}\\
      =\\
        & e^{-\lambda}\,e^\lambda\\
      =\\
        & 1
    \end{longderivation}
    \item~
    \begin{longderivation}<1>
        & \sum_{n=0}^\infty x\frac{\lambda^xe^{-\lambda}}{x!}\\
      =\\
        & \sum_{n=1}^\infty\frac{\lambda^xe^{-\lambda}}{(x-1)!}\\
      =\\
        & \lambda e^{-\lambda}\sum_{x=0}^\infty\frac{\lambda^x}{x!}\\
      =\\
        & \lambda
    \end{longderivation}
    \item~
    \begin{longderivation}<1>
        & \sum_{x=0}^\infty x^2\frac{\lambda^xe^{-\lambda}}{x!} - \lambda^2\\
      =\\
        & \sum_{n=1}^\infty x\frac{\lambda^xe^{-\lambda}}{(x-1)!} - \lambda^2\\
      =\\
        & \lambda\sum_{n=0}^\infty x\frac{\lambda^xe^{-\lambda}}{x!}
        + \lambda\sum_{n=0}^\infty \frac{\lambda^xe^{-\lambda}}{x!}
        - \lambda^2\\
      =\\
        & \lambda^2 + \lambda - \lambda^2\\
      =\\
        & \lambda
    \end{longderivation}
  \end{enumerate}

  Para finalizar, se mostrará la validez de los procedimientos usados para
  estos cálculos. Se afirma que la serie $\sum_x \left|x^t \frac{\lambda^x}{x!}\right|$
  converge para todo $t\in\R$. 
  Por el criterio de la razón, cuando $x\to\infty$,
  \[
    \left|(x+1)^t\frac{\lambda^{x+1}e^\lambda}{(x+1)!}\,
    \frac{x!}{x^t\lambda^x}\right|
    =
    \left(1 + \frac{1}{x}\right)^t\,\frac{\lambda}{x+1}
    \to 0 < 1
  \]
  La convergencia absoluta de la serie permite el reordenamiento de la misma y
  la separación en sumas.
\end{Demo}
\input{Discretas/Geométrica.tex}
\clearpage
\subsection{Distribuciones Continuas}
Las distribuciones continuas son aquellas cuyas funciones de
densidad tienen dominio en un conjunto denso en si mismo, el cual
en esta sección se asumirá como $\R$. De no ser especificado
el valor de una función de densidad en algún subconjunto de $\R$,
este se asumirá como $0$.
\subsubsection{Distribución Normal}
\begin{Def}
  Sea $X$ una variable aleatoria continua. $X$ tiene distribución normal
  de parámetros $\mu\in\R$, $\sigma^2\in\R^+$ cuando su función de densidad es
  \[
    f(x) = \dfrac{1}{\sqrt{2\pi\sigma^2}}
    e^{-\nicefrac{(x-\mu)^2}{\left(2\sigma^2\right)}}
    \qquad (x\in\R)
  \]
  Esto se denotará como $X\sim N(\mu,\sigma^2)$.
\end{Def}
\begin{Teo}
  Sean $X\sim N(\mu,\sigma^2)$ y $f$ la función de masa de $X$. Entonces,
  \begin{enumerate}
    \item Para todo $x\in\R$, $f(x) \geq 0$.
    \item $\int_{\R}{f(x)}\diff{x} = 1$.
    \item $\text{E}[X] = \mu$.
    \item $\text{Var}[X] = \sigma^2$.
  \end{enumerate}
\end{Teo}

Para las integrales involucradas en los resultados del teorema es
conveniente tener el siguiente resultado anterior a proceder con el teorema.
\[I = \int_\R e^{-x^2}\diff{x} = \sqrt{\pi}\]
\begin{longderivation}
    & I^2\\
  =\\
    & \left(\int_{\R}e^{-x^2}\diff{x}\right)\left(\int_{\R}e^{-x^2}\diff{x}\right)\\
  =\\
    & \left(\int_{\R}e^{-x^2}\diff{x}\right)\left(\int_{\R}e^{-y^2}\diff{y}\right)\\
  =\\
    & \int_{\R}\int_{\R}e^{-(x^2+y^2)}\diff{x,y}\\
  =\\
    & \int_{\R^2}e^{-(x^2+y^2)}\diff{x,y}\\
  =\\
    & \smashoperator[r]{\int_{\R^2-\set{(x,0)\in\R^2}{x\leq0}}}
    e^{-(x^2+y^2)}\diff{x,y}
\end{longderivation}
La validez de este último paso se justifica demostrando que la integral de esta función
sobre el conjunto $\set{(x,0)\in\R^2}{x\leq0}$ es nula.
Esto se hará en dos momentos.

Inicialmente, se mostrará
que la integral sobre el conjunto $\{(0,0)\}$ es nula.
Para $n\in\Z^+$, se define la sucesión $K_n=[-\nicefrac{1}{n},\nicefrac{1}{n}]$. Para todo
$n\in\Z^+$, $K_n$ es compacto y $K_{n+1}\subseteq K_n$, luego $\bigcap_{n\in\Z^+}K_n\not=\varnothing$
y $\diam\bigcap_{n\in\Z^+}K_n = 0$. Dado que, para todo $n\in\Z^+$, se
tiene que $0\in K_n$, entonces, $\bigcap_{n\in\Z^+}K_n=\{0\}$.

Como el máximo de $\left|e^{-(x^2+y^2)}\right|$ es $1$, y la longitud del contorno $K_n\times K_n$ es
$\nicefrac{4}{n}$, entonces,
\[
  \left|\smashoperator[r]{\int_{K_n\times K_n}}
  e^{-(x^2+y^2)}\diff{x,y}\right|\leq \frac{4}{n}
\]

Con esto se obtiene entonces que
\[
  \int_{\{(0,0)\}}e^{-(x^2+y^2)}\diff{x,y} = 0
\]

Con esto, la integral sobre $\set{(x,0)}{x\leq0}$ coincide con la interal sobre
$\set{(x,0)}{x<0}=\R^-\times\{0\}$. Para $n\in\Z^+$, se define la sucesión
\[
  S_n = \set{(x,y)\in\R^2}{
    x<0\land\arctan\left|\frac{y}{x}\right|\leq\frac{1}{n}
  }\cup\{0,0\}
\]

Por definición, se obtiene que
\begin{longderivation}
    & (x,y)\in S_n\\
  \iff\\
    & (x,y)=(0,0) \lor
    \Forall{n}[n\in\Z^+]{
      x<0\land\arctan\left|\frac{y}{x}\right|\leq\frac{1}{n}
    }
\end{longderivation}
De esta última proposición, se obtiene que
\[\left(R^-\times\{0\}\right)\cup\{0,0\}\subseteq\bigcap_{n\in\Z^+}S_n\]
Ashora, supóngase que existe $(x,y)\in\R^2$ tal que
$(x,y)\in\bigcap_{n\in\Z^+}S_n-
\left(\left(R^-\times\{0\}\right)\cup\{0,0\}\right)$.
Es decir,
\[
  \Forall{n}[n\in\Z^+]{
      x<0\land\arctan\left|\frac{y}{x}\right|\leq\frac{1}{n}
    }
  \land
  y\not=0
\]
Lo cual es contradictorio, pues de ser el caso $a = \arctan\left|\frac{y}{x}\right| >0$,
pero existe $N\in\Z^+$ tal que $a > \frac{1}{N}$. Así, se concluye la igualdad
\[\bigcap_{n\in\Z^+}S_n = \left(R^-\times\{0\}\right)\cup\{(0,0)\}\]

Ahora, para mostrar que la integral sobre este conjunto es nula, se
integrará sobre $S_n$. $S_n$, por el resultado previo, puede
ser integrado sin tomar en cuenta $\{(0,0)\}$. De esta forma,
una parametrización de $S_n-\{(0,0)\}$ es
\[\vv{t(r,\theta)}=\left<r\cos(\theta),r\sin(\theta)\right>
\qquad(r,\theta)\in\R^+\times[\pi-\nicefrac{1}{n},\pi+\nicefrac{1}{n}]\]
acomodando $\vv{t}$ para $\R^3$, el elemento de área es
\begin{longderivation}<0.9>
    & \left|\vv{t_r} \times \vv{t_\theta}\right|\\
  =\\
    & \left|
      \left<\cos(\theta),\sin(\theta),0\right>
      \times
      \left< -r\sin(\theta),r\cos(\theta),0\right>
    \right|\\
  =\\
    & \left|
      \left<0,0,r\cos^2(\theta) + r\sin^2(\theta)\right>
    \right|\\
  =\\
    & r
\end{longderivation}
Por el resutlado anterior a cerca de la integral sobre $\{(0,0)\}$, se puede
extender el dominio de $\vv{t}$ de forma que se pierda la biyección únicamente
en $\{(0,0)\}$. Así, tomando $r\in\R^+\cup\{0\}$, la integral sobre $S_n$
resulta igual a
\begin{longderivation}
    & \int_0^{\infty}\int_{\pi-\nicefrac{1}{n}}^{\pi+\nicefrac{1}{n}}
    re^{-r^2}\diff{\theta,r}\\
  =\\
    & \frac{1}{n}\int_0^{\infty}2re^{-r^2}\diff{r}\\
  \why[=]{Tomando $u=r^2$}\\
    & \frac{1}{n}
\end{longderivation}
Así,
\[
\smashoperator{\int_{\set{(x,0)\in\R^2}{x\leq0}}}
e^{-\left(x^2+y^2\right)}\diff{x,y}
=
\lim_{n\to\infty}\int_{S_n}e^{-\left(x^2+y^2\right)}\diff{x,y}
=
\lim_{n\to\infty}\frac{1}{n}
= 0
\]

Procediendo ahora sí con la integral sobre $\R^2-\set{(x,0)\in\R^2}{x\leq0}$,
se utilizará la misma parametrización que se mostró para el último
resultado, únicamente cambiando su dominio por
\[(r,\theta)\in\R^+\times(-\pi,\pi)\]
Por el resutlado anterior, se puede extender el dominio a
$\left(R^+\cup\{0\}\right)\times[-\pi,\pi]$.
Así, $I^2$ resulta igual a
\begin{longderivation}
    & \int_0^\infty\int_{-\pi}^\pi re^{-r^2}\diff{\theta,r}\\
  =\\
    & 2\pi\int_0^\infty re^{-r^2}\diff{r}\\
  \why[=]{Tomando $u=r^2$}\\
    & \pi\int_0^\infty e^{-u}\diff{u}\\
  =\\
    & \pi
\end{longderivation}
Con lo que, $I = \sqrt{\pi}$.

Continuando con la demostración del teorema
\begin{Demo}
  Para las integrales implicadas en este resultado, se utilizará eventualmente el
  siguiente cambio de variable
  \[t = \frac{x - \mu}{\sqrt{2\sigma^2}}\]
  Cuando $x\to\pm\infty$, $t\to\pm\infty$.
  \begin{enumerate}
    \item Dado que todos los términos de la función de densidad son no negativos, se cumple.
    \item~
    \begin{longderivation}
        & \int_{-\infty}^{\infty}\dfrac{1}{\sqrt{2\pi\sigma^2}}
        e^{-\nicefrac{(x-\mu)^2}{\left(2\sigma^2\right)}}\diff{x}\\
      \why[=]{Haciendo uso del cambio de variable presentado}\\
        & \frac{1}{\sqrt{\pi}}\int_{-\infty}^{\infty}e^{-t^2}\diff{t}\\
      =\\
        & 1
    \end{longderivation}
    \item~
    \begin{longderivation}
        & \text{E}[X]\\
      =\\
        & \int_{-\infty}^{\infty}\dfrac{1}{\sqrt{2\pi\sigma^2}}
        x\,e^{-\nicefrac{(x-\mu)^2}{\left(2\sigma^2\right)}}\diff{x}\\
      \why[=]{Haciendo uso del cambio de variable presentado}\\
        & \dfrac{1}{\sqrt{\pi}}\int_{-\infty}^{\infty}
        (\sqrt{\pi\sigma^2}t + \mu)e^{-t^2}\diff{t}\\
      =\\
        & \sqrt{\frac{2\sigma^2}{\pi}}\int_{-\infty}^{\infty}te^{-t^2}\diff{t}
        + \frac{\mu}{\sqrt{\pi}}\int_{-\infty}^{\infty}e^{-t^2}\diff{t}\\
      =\\
        & \sqrt{\frac{2\sigma^2}{\pi}}\left(
          \lim_{R_1\to\infty}\int_{-R_1}^0te^{-t^2}\diff{t}
          + \lim_{R_2\to\infty}\int_0^{R_2}te^{-t^2}\diff{t}
        \right) + \mu
    \end{longderivation}
    Para proceder con ambas integrales, se realiza el cambio de variable
    $u=x^2$, cuando $x=0$, $u=0$, cuando $x=\pm R_{1,2}$, $u=R_{1,2}^2$. Así
    \begin{longderivation}
        & \sqrt{\frac{2\sigma^2}{\pi}}\left(
            \lim_{R_1\to\infty}\int_{-R_1}^0te^{-t^2}\diff{t}
            + \lim_{R_2\to\infty}\int_0^{R_2}te^{-t^2}\diff{t}
          \right) + \mu\\
      =\\
        & \sqrt{\frac{2\sigma^2}{\pi}}\left(
          \lim_{R_1\to\infty}-\int_0^{R_1}e^{-u}\diff{u}
          + \lim_{R_2\to\infty}\int_0^{R_2}e^{-u}\diff{u}
        \right) + \mu\\
      =\\
        & \sqrt{\frac{2\sigma^2}{\pi}}\left(
          \lim_{R_1\to\infty} -1 + e^{-R_1}
          + \lim_{R_2\to\infty} -e^{-R_2} + 1
        \right) + \mu\\
      =\\
        & \mu
    \end{longderivation}
    \item~
    \begin{longderivation}
        & \text{Var}[X]\\
      =\\
        & \text{E}[X^2] - \text{E}[X]\\
      =\\
        & \int_{-\infty}^{\infty}\dfrac{1}{\sqrt{2\pi\sigma^2}}
        x^2e^{-\nicefrac{(x-\mu)^2}{\left(2\sigma^2\right)}}\diff{x}
        - \mu^2\\
      \why[=]{Haciendo uso del cambio de variable presentado}\\
        & \frac{1}{\sqrt{\pi}}\int_{-\infty}^{\infty}
        (\sqrt{2\sigma^2}t + \mu)^2e^{-t^2}\diff{t} - \mu^2\\
      =\\
        & \frac{1}{\sqrt{\pi}}\left(
          2\sigma^2\int_{-\infty}^{\infty}t^2e^{-t^2}\diff{t}
          + 2\sqrt{2\sigma^2}\mu\int_{-\infty}^{\infty}te^{-t^2}\diff{t}
          + \mu^2\int_{-\infty}^{\infty}e^{-t^2}\diff{t}
        \right) - \mu^2\\
      =\\
        & \frac{2\sigma^2}{\sqrt{\pi}}\int_{-\infty}^{\infty}
        t^2e^{-t^2}\diff{t}\\
      =\\
        & \frac{2\sigma^2}{\sqrt{\pi}}\left(
          \lim_{R_1\to\infty}\int_{-R_1}^0 t^2e^{-t^2}\diff{t}
          + \lim_{R_2\to\infty}\int_0^{R_2}t^2e^{-t^2}\diff{t}
        \right)
    \end{longderivation}
    Nótese que $\odv*{e^{-t^2}}{t}=-2te^{-t^2}$, con lo que
    resulta conveniente hacer integración por partes tomando
    una de las funciones como $t$ y la otra como $te^{-t^2}$.
    Así,
    \begin{longderivation}
        & \frac{2\sigma^2}{\sqrt{\pi}}\left(
            \lim_{R_1\to\infty}\int_{-R_1}^0 t^2e^{-t^2}\diff{t}
            + \lim_{R_2\to\infty}\int_0^{R_2}t^2e^{-t^2}\diff{t}
          \right)\\
      =\\
        & \frac{2\sigma^2}{\sqrt{\pi}}\left(
          \lim_{R_1\to\infty}
          R_1\frac{e^{-R_1^2}}{2} + \frac{1}{2}\int_{-R_1}^0e^{-t^2}\diff{t}
          + \lim_{R_2\to\infty}
          -R_2\frac{e^{-R_2^2}}{2} + \frac{1}{2}\int_0^{R_2}e^{-t^2}\diff{t}
        \right)\\
      =\\
        & \frac{2\sigma^2}{\sqrt{\pi}}\frac{1}{2}\int_{-\infty}^{\infty}
        e^{-t^2}\diff{t}\\
      =\\
        & \sigma^2
    \end{longderivation}
  \end{enumerate}
\end{Demo}

\subsubsection{Distribución Gamma}
\begin{Def}
  Sea $X$ una variable aleatoria continua. $X$ tiene distribución Gamma
  de parámetros $\alpha,\beta\in\R^+$, cuando su función de densidad es
  \[
    f(x)=\frac{\beta^{\alpha}}{\Gamma(\alpha)}x^{\alpha-1}e^{-\beta x}
    \qquad (x\in\R^+)
  \]
\end{Def}
\begin{Teo}
  Sean $X$ una variable aleatoria continua con distribución Gamma
  de parámetros $\alpha,\beta$ y $f$
  su función de densidad. Entonces,
  \begin{enumerate}
    \item Para todo $x\in\R$, $f(x) \geq 0$.
    \item $\int_{\R}f(x)\mathrm{d}x=1$.
    \item $\text{E}[X]=\frac{\alpha}{\beta}$
    \item $\text{Var}[X]=\frac{\alpha}{\beta^2}$
  \end{enumerate}
\end{Teo}
\begin{Demo}
  Para las integrales implicadas en este resultado, se utilizará eventualmente
  el siguiente cambio de variable
  \[t = \beta x\]
  Cuando $x=0$, $t=0$ y cuando $x\to\infty$, $t\to\infty$.
  \begin{enumerate}
    \item Todos los términos en la definición de $f$ son no negativos, con lo que
    $f(x)\geq0$.
    \item~
    \begin{longderivation}
        & \int_0^{\infty}
        \frac{\beta^{\alpha}}{\Gamma(\alpha)}x^{\alpha-1}e^{-\beta x}
        \diff{x}\\
      \why[=]{Haciendo uso del cambio de variable presentado}\\
        & \frac{\beta^{\alpha}}{\Gamma(\alpha)}\frac{1}{\beta}
        \int_0^{\infty}\left(\frac{t}{\beta}\right)^{\alpha-1}e^{-t}
        \diff{t}\\
      =\\
        & \frac{1}{\Gamma(\alpha)}
        \int_0^{\infty}t^{\alpha-1}e^{-t}\diff{t}\\
      =\\
        & \frac{1}{\Gamma(\alpha)}\Gamma(\alpha)\\
      =\\
        & 1
    \end{longderivation}
    \item~
    \begin{longderivation}
      & \text{E}[X]\\
    =\\
      & \int_0^{\infty}
      \frac{\beta^{\alpha}}{\Gamma(\alpha)}x^{\alpha}e^{-\beta x}
      \diff{x}\\
    \why[=]{Haciendo uso del cambio de variable presentado}\\
      & \frac{\beta^{\alpha}}{\Gamma(\alpha)}\frac{1}{\beta}
      \int_0^{\infty}\left(\frac{t}{\beta}\right)^{\alpha}e^{-t}
      \diff{t}\\
    =\\
      & \frac{1}{\beta\Gamma(\alpha)}
      \int_0^{\infty}t^{\alpha}e^{-t}\diff{t}\\
    =\\
      & \frac{1}{\beta\Gamma(\alpha)}\Gamma(\alpha+1)\\
    =\\
      & \frac{\alpha\Gamma(\alpha)}{\beta\Gamma(\alpha)}\\
    =\\
      & \frac{\alpha}{\beta}
  \end{longderivation}
    \item~
    \begin{longderivation}
      & \text{Var}[X]\\
    =\\
      & \text{E}[X^2] - \text{E}^2[X]\\
    =\\
      & \int_0^{\infty}
      \frac{\beta^{\alpha}}{\Gamma(\alpha)}x^{\alpha+1}e^{-\beta x}
      \diff{x} - \frac{\alpha^2}{\beta^2}\\
    \why[=]{Haciendo uso del cambio de variable presentado}\\
      & \frac{\beta^{\alpha}}{\Gamma(\alpha)}\frac{1}{\beta}
      \int_0^{\infty}\left(\frac{t}{\beta}\right)^{\alpha+1}e^{-t}
      \diff{t} - \frac{\alpha^2}{\beta^2}\\
    =\\
      & \frac{1}{\beta^2\Gamma(\alpha)}
      \int_0^{\infty}t^{\alpha+1}e^{-t}\diff{t}
      - \frac{\alpha^2}{\beta^2}\\
    =\\
      & \frac{1}{\beta^2\Gamma(\alpha)}\Gamma(\alpha+2)
      - \frac{\alpha^2}{\beta^2}\\
    =\\
      & \frac{\alpha(\alpha+1)\Gamma(\alpha)}{\beta^2\Gamma(\alpha)}
      - \frac{\alpha^2}{\beta^2}\\
    =\\
      & \frac{\alpha(\alpha+1)}{\beta^2} - \frac{\alpha^2}{\beta^2}\\
    =\\
      & \frac{\alpha}{\beta^2}
  \end{longderivation}
  \end{enumerate}
\end{Demo}
\subsubsection{Chi-cuadrada}
\begin{Def}
    Sea $X$ una variable aleatoria continua, $X$ tiene una
    distribución $\chi^2$ con parámetro $v\in\R^+$ si su función de densidad
    es:
    \[
        f(x) = \frac{1}{2^{\nicefrac{v}{2}}\Gamma\left(\frac{v}{2}\right)}
        x^{\left(\nicefrac{v}{2}\right)-1}e^{\nicefrac{-x}{2}} \qquad(x\geq0)
    \]
    Esto se denotará como $X\sim\chi^2(v)$.
\end{Def}

\begin{Teo}
    Sean $X \sim \chi^2(v)$ y $f$ su función de densidad. Entonces:
    \begin{enumerate}
        \item para todo $x\in \R$, $f(x)\geq0$
        \item $\int_{0}^{\infty}f(x)\diff{x}=1$
        \item $\text{E}[X]= v$
        \item $\text{Var}[X]= 2v$
    \end{enumerate}
\end{Teo}

\begin{Demo}~
    \begin{enumerate}
        \item Dado que todos los factores del término que componen a
        $f$ son no negativos, $f(x)\geq0$ para todo $x\geq 0$.
        \item Para demostrar esto, basta con demostrar que 
        \[
            \int_{0}^{\infty}x^{\left(\nicefrac{v}{2}\right)-1}
            e^{\nicefrac{-x}{2}}\diff{x}
            =2^{\nicefrac{v}{2}}\Gamma\left(\frac{v}{2}\right)
        \]
            El procedimiento inicia con el cambio de variable
            $u=\displaystyle\frac{x}{2}$. Si $x=0$, $u=0$ 
            y si $x\to\infty$, $u\to\infty$. De la misma 
            manera $\diff{u} = \displaystyle\frac{1}{2}\diff{x}$, por lo 
            tanto $\diff{x}=2\diff{u}$, así:
        \begin{center}
            \begin{longderivation}
                & \int_{0}^{\infty}x^{(\nicefrac{v}{2})-1}
                e^{\nicefrac{-x}{2}}\diff{x}\\
                =\\
                & \int_{0}^{\infty}(2u)^{(\nicefrac{v}{2})-1}e^{-u}2\diff{u}\\
                =\\
                & 2^{\nicefrac{v}{2}}\int_{0}^{\infty}
                u^{(\nicefrac{v}{2})-1}e^{-u}\diff{u}
            \end{longderivation}
        \end{center}
        Recordando que la definición de la función $\Gamma$ es 
        \[
            \Gamma(x)=\int_{0}^{\infty}t^{x-1}e^{-t}\diff{t}
        \]
        Por lo que se obtiene:
        \[
            \int_{0}^{\infty}x^{(\nicefrac{v}{2})-1}
            e^{\nicefrac{-x}{2}}\diff{x} =
            2^{\nicefrac{v}{2}}\Gamma\left(\frac{v}{2}\right)
        \]
        Que es lo que se quería demostrar.
        \item Para empezar la demostración, se hará la misma sustitución
        del item pasado.

        \begin{center}
            \begin{longderivation}
                & \text{E}[X]\\
                =\\
                & \int_{-\infty}^{\infty} xf(x)dx\\
                =\\
                & \frac{1}{2^{\nicefrac{v}{2}}
                \Gamma\left(\frac{v}{2}\right)}\int_{0}^{\infty}x^{
                \nicefrac{v}{2}}e^{\nicefrac{-x}{2}}\diff{x}\\
                \why[=]{\text{Cambio de variable}}\\
                & \frac{1}{2^{\nicefrac{v}{2}}
                \Gamma\left(\frac{v}{2}\right)}\int_{0}^{\infty}(2u)^{
                \nicefrac{v}{2}}e^{-u}2\diff{u}\\
                =\\
                & \frac{2}{\Gamma\left(\frac{v}{2}\right)}
                \int_{0}^{\infty}u^{\nicefrac{v}{2}}e^{-u}\diff{u}\\
                \why[=]{\text{Definición de $\Gamma$}}\\
                & \frac{2}{\Gamma\left(\frac{v}{2}\right)}
                \Gamma\left(\frac{v}{2}+1\right)
            \end{longderivation}
        \end{center}
        Por propiedad de la función $\Gamma$, $\Gamma(x+1)=x\Gamma(x)$, 
        la igualdad se transforma a lo siguiente:
        \begin{center}
            \begin{longderivation}
                &\text{E}[X]=\left(\frac{2}
                {\Gamma\left(\frac{v}{2}\right)}\right)\left(\frac{v}
                {2}\right)\Gamma\left(\frac{v}{2}\right)\\
                \iff\\
                &\text{E}[X]=v
            \end{longderivation}
        \end{center}
        Lo que demuestra la propiedad
        \item Nuevamente se utiliza el mismo cambio de variable para esta demostración.
        \begin{center}
            \begin{longderivation}
                & \text{Var}[X]\\
                =\\
                & \text{E}[X^2]-\text{E}^2[X]\\
                =\\
                &\frac{1}{2^{\nicefrac{v}{2}}\Gamma\left(\frac{v}{2}\right)}
                \int_{0}^{\infty}x^{2}x^{(\nicefrac{v}{2})-1}
                e^{\nicefrac{-x}{2}}\diff{x} -v^2\\
                =\\
                &\frac{1}{2^{\nicefrac{v}{2}}\Gamma\left(\frac{v}{2}\right)}
                \int_{0}^{\infty}x^{(\nicefrac{v}{2})+1}
                e^{-\nicefrac{x}{2}}\diff{x} - v^2\\
                \why[=]{\text{Cambio de variable}}\\
                &\frac{1}{2^{\nicefrac{v}{2}}\Gamma\left(\frac{v}{2}\right)}
                \int_{0}^{\infty}(2u)^{(\nicefrac{v}{2})+1}
                e^{-u}2\diff{u}-v^2\\
                =\\
                &\frac{1}{2^{\nicefrac{v}{2}}\Gamma\left(\frac{v}{2}\right)}
                2^{(\nicefrac{v}{2})+2}
                \int_{0}^{\infty}u^{(\nicefrac{v}{2})+1}e^{-u}\diff{u}-v^2\\
                \why[=]{\text{Definición de $\Gamma$}}\\
                &\frac{4}{\Gamma\left(\frac{v}{2}\right)}\Gamma
                \left(\frac{v}{2}+2\right)-v^2\\
                \why[=]{\text{propiedades de $\Gamma$}}\\
                &\frac{4}{\Gamma\left(\frac{v}{2}\right)}\left
                (\left(\frac{v}{2}+1\right)\left
                (\frac{v}{2}\right)\Gamma\left(\frac{v}{2}\right)\right)-v^2\\
                =\\
                &4\left(\frac{v+2}{2}\right)\left(\frac{v}{2}\right)-v^2\\
                =\\
                &(v+2)(v)-v^2\\
                =\\
                &v^2+2v-v^2\\
                =\\
                2v
            \end{longderivation}
        \end{center}    
        Así, se obtiene que Var$[X]=2v$, lo que concluye la demostración.     
    \end{enumerate}
\end{Demo}
\subsubsection{Distribución t}
\begin{Def}
    Sea $T$ una variable aleatoria continua. $T$ tiene distribución $t$
    de parámetro $v \in \R^+$ cuando su función de densidad es:
    \[
        f(x)=\frac{\Gamma\left(\frac{v+1}{2}\right)}
        {\sqrt{v\pi}\Gamma\left(\frac{v}{2}\right)}
        \left(1+\frac{x^2}{v}\right)^{-\nicefrac{(v+1)}{2}}
        =
        \frac{1}{\sqrt{v}B\left(\frac{1}{2}\text{,}\frac{v}{2}\right)}
        \left(1+\frac{x^2}{v}\right)^{-\nicefrac{(v+1)}{2}}
        \qquad (x \in \R)
    \]
    Esto se va a denotar como $T\sim t(v)$
\end{Def}
\begin{Teo}
    Sea $T\sim t(v)$ y $f$ su función de densidad. Entonces:
    \begin{enumerate}
        \item Para todo $x\in\R$, $f(x) \geq 0$.
        \item $\int_{\R}f(x)\diff{x}=1$.
        \item $\text{E}[T]=0$
        \item $\text{Var}[T]=\frac{v}{v-2}$
    \end{enumerate}
\end{Teo}
\begin{Demo}~
    \begin{enumerate}
        \item Dado que $\Gamma(x)\geq 0$ para todo $x$, y 
        $(1+x^2)^v \geq 0$ para todo $x$ y $v\geq 0$, entonces
        $f(x)\geq 0$ para todo $x$

        \item Para demostrarlo, basta con demostrar que:
        \[
            \int_{-\infty}^{\infty} \left(1+\frac{x^2}{v}\right)^   
            {-\nicefrac{(v+1)}{2}}\diff{x}
            =
            \sqrt{v}B\left(\frac{1}{v}\text{,}\frac{v}{2}\right)
        \]
        Dado que $f$ es una función par
        \[
            \int_{-\infty}^{\infty}\left(1+\frac{x^2}{v}\right)^
            {-\nicefrac{(v+1)}{2}}\diff{x}
            =2\int_{0}^{\infty}\left(1+\frac{x^2}{v}\right)^
            {-\nicefrac{(v+1)}{2}}\diff{x}
        \]
        Haciendo la sustitución $\frac{x}{v}=\tan(\theta)$. Cuando
        $x=0$, $\theta=0$ y cuando $x \to \infty$, $\theta \to \frac{\pi}{2}$. Como 
        $\odv*{\tan(\theta)}{\theta}=\sec^2(\theta)$ la expresión resulta igual a
        \begin{longderivation}
            &2\sqrt{v}\int_{0}^{\frac{\pi}{2}}(1+\tan^2(\theta))^{-\nicefrac{(v+1)}{2}}
            \sec^2(\theta)\diff{\theta}\\
            =\\
            &2\sqrt{v}\int_{0}^{\frac{\pi}{2}}(\sec^2(\theta))^{-\nicefrac{(v+1)}{2}}
            \sec^2(\theta)\diff{\theta}\\
            =\\
            &2\sqrt{v}\int_{0}^{\frac{\pi}{2}}
            (\sec(\theta))^{-v+1}\diff{\theta}\\
            =\\
            &2\sqrt{v}\int_{0}^{\frac{\pi}{2}}
            (\cos(\theta))^{v-1}\diff{\theta}
        \end{longderivation}
        Haciendo la sustitución $u=\sin(\theta)$. Cuando
        $\theta=0$, $u=0$ y cuando $\theta = \frac{\pi}{2}$, $u=1$. Como 
        $\odv*{\sin(\theta)}{\theta}=\cos(\theta)$ y $\cos(\theta) = 
        (1-u^2)^{\nicefrac{1}{2}}$ cuando $\left(0 \leq \theta \leq \frac{\pi}{2}\right)$, 
        la expresión resulta igual a
        \[
            2\sqrt{v}\int_{0}^{1}(1-u^2)^{(\nicefrac{v}{2})-1}\diff{u}
        \]
        Y como última sustitución, $t=u^2$. Cuando $u=1$, $t=1$ y cuando
        $u=0$, $t=0$. Como $\odv*{u^2}{u}=2u$ entonces $\diff{u} = 
        \frac{\diff{t}}{2\sqrt{t}}$, y la expresión resultante es
        \begin{longderivation}
            &\sqrt{v}\int_{0}^{1}t^{\nicefrac{1}{2}-1}
            (1-t)^{\nicefrac{v}{2}-1}\diff{t}\\
            =\\
            &\sqrt{v}B\left(\frac{1}{2}\text{,}\frac{v}{2}\right)
        \end{longderivation} 
        \item Dado que $f(x)=x(1+\frac{x^2}{v})^{-\nicefrac{(v+1)}{2}}$
        es un función impar 
        \begin{longderivation}
            &\lim_{R\to\infty}\int_{-R}^{R}f(x)\diff{x}\\
            =\\
            &\lim_{R\to\infty}\int_{-R}^{0}f(x)\diff{x}
            + \int_{0}^{R}f(x)\diff{x}\\
            =\\
            &\lim_{R\to\infty}-\int_{R}^{0}f(x)\diff{x}
            +\int_{R}^{0}f(x)\diff{x}\\
            =\\
            &0
        \end{longderivation}
        \item Denotando
        \[
            A = \frac{\Gamma\left(\frac{v+1}{2}\right)}
            {\sqrt{v\pi}\Gamma\left(\frac{v}{2}\right)}
        \]
        Se van a hacer las mismas secuencias de sustituciones que
        en el item (ii)
        \begin{longderivation}
            &\text{Var}[X]\\
            =\\
            &\int_{-\infty}^{\infty}f(x)\diff{x}\\
            =\\
            &2A \int_{0}^{\infty}x^2\left(1+\frac{x^2}{v}\right)^
            {-\nicefrac{(v+1)}{2}}\\
            \why[=]{$\frac{x}{\sqrt{v}}=\tan(\theta)$}\\
            &2vA\sqrt{v}\int_{0}^{\frac{\pi}{2}}
            \tan^2(\theta)(1+\tan^2(\theta))^{-\nicefrac{(v+1)}{2}}
            \sec^2(\theta)\diff{\theta}\\
            =\\
            &2Av^{\nicefrac{3}{2}}\int_{0}^{\frac{\pi}{2}}
            \tan^2(\theta)\cos^{v-1}(\theta)\diff{\theta}\\
            =\\
            &2Av^{\nicefrac{3}{2}}\int_{0}^{\frac{\pi}{2}}
            \sin^2(\theta)\cos^{v-3}(\theta)\diff{\theta}\\
            \why[=]{$u=\sin(\theta)$}\\
            &2Av^{\nicefrac{3}{2}}\int_{0}^{1}
            u^2(1-u^2)^{\nicefrac{v-4}{2}}\diff{u}\\
            \why[=]{$u^2=t$}\\
            &Av^{\nicefrac{3}{2}}\int_{0}^{1}
            t^{\nicefrac{1}{2}}(1-t)^{(\nicefrac{v}{2})-2}\diff{t}\\
            \why[=]{Expansión de A}\\
            &\frac{\Gamma\left(\frac{v+1}{2}\right)}
            {\sqrt{v\pi}\Gamma\left(\frac{v}{2}\right)}
            \frac{\Gamma\left(\frac{3}{2}\right)\Gamma\left(\frac{v}{2}-1\right)}
            {\Gamma\left(\frac{v+1}{2}\right)}v^{\nicefrac{3}{2}}\\
            =\\
            &\frac{v}{2\sqrt{\pi}}
            \frac
            {\Gamma\left(\frac{1}{2}\right)\Gamma\left(\frac{v}{2}-1\right)}
            {\left(\frac{v}{2}-1\right)\Gamma\left(\frac{v}{2}-1\right)}\\
            =\\
            &\frac{v\sqrt{\pi}}{2\sqrt{\pi}\left(\frac{v-2}{2}\right)}\\
            =\\
            &\frac{\frac{v}{2}}{\frac{v-2}{2}}\\
            =\\
            &\frac{v}{v-2}
        \end{longderivation}
        Con lo que finaliza la demostración.
    \end{enumerate}
\end{Demo}
\subsubsection{Distribucion F}

\begin{Def}
  Sean $F$ una variable aleatoria continua. $F$ tiene una distribución
  $\mathbf{f}$ de parámetros $u,v\in\R^+$, cuando su función de densidad es
  \[f(x)=
  \dfrac{\Gamma\left(\frac{u+v}{2}\right)\left(\frac{u}{v}\right)^{\nicefrac{u}{2}}}
  {\Gamma\left(\frac{u}{2}\right)\Gamma\left(\frac{v}{2}\right)}
  \dfrac{x^{(\nicefrac{u}{2})-1}}
  {\left(1 + \frac{u}{v}x\right)^{\nicefrac{(u+v)}{2}}}
  =
  \dfrac{\left(\frac{u}{v}\right)^{\nicefrac{u}{2}}}
    {B\left(\frac{u}{2},\frac{v}{2}\right)}
  \dfrac{x^{(\nicefrac{u}{2})-1}}
    {\left(1 + \frac{u}{v}x\right)^{\nicefrac{(u+v)}{2}}}
  \qquad (x\in\R^+)
  \]
  Esto se denotará como $F\sim\mathbf{f}(u,v)$.
\end{Def}
\begin{Teo}
  Sean $F\sim \mathbf{f}(u,v)$ y $f$ su función de densidad. Entonces,
  \begin{enumerate}
    \item Para todo $x\in\R$, $f(x) \geq 0$.
    \item $\int_{\R}f(x)\mathrm{d}x=1$.
    \item $\text{E}[F]=\frac{v}{v-2} \qquad(v>2)$
    \item $\text{Var}[F]=\frac{2v^2(u+v-2)}{u(v-2)^2(v-4)}\qquad(v>4)$
  \end{enumerate}
\end{Teo}
\begin{Demo}~
  \begin{enumerate}
    \item Todos los términos de la expresión que define $f$ son no negativos para
    $x\in\R^+$, con lo que se concluye $f(x)\geq0$.
    \item Basta mostrar la siguiente igualdad
    \[
      \int_0^\infty{
      \dfrac{x^{(\nicefrac{u}{2})-1}}
      {\left(1 + \frac{u}{v}x\right)^{\nicefrac{(u+v)}{2}}}
      }\diff{x}
      =
      \left(\frac{u}{v}\right)^{-\nicefrac{u}{2}}
      B\left(\frac{u}{2},\frac{v}{2}\right)
    \]
    Recordando que
    \[
      B\left(\frac{u}{2},\frac{v}{2}\right) =
      \int_0^1{t^{(\nicefrac{u}{2})-1}(1-t)^{(\nicefrac{v}{2})-1}}\diff{t}
    \]

    El procedimiento inicia con el siguiente cambio de variable:
    \[\tan^2(\theta)=\frac{u}{v}x\]
    Cuando $x=0$, $\theta=0$ y cuando $x\to\infty$, $\theta\to\nicefrac{\pi}{2}$.
    Recordando que $\odv*{\tan^2(\theta)}{\theta}=2\tan(\theta)\sec^2(\theta)$,
    se obtienen las siguientes igualdades
    \begin{longderivation}
        & \int_0^\infty{
          \dfrac{x^{(\nicefrac{u}{2})-1}}
          {\left(1 + \frac{u}{v}x\right)^{\nicefrac{(u+v)}{2}}}
          }\diff{x}\\
      =\\
        & 2\frac{v}{u}\int_0^{\nicefrac{\pi}{2}}{
          \dfrac{\left(\frac{v}{u}\right)^{(\nicefrac{u}{2})-1}\tan^{u-2}(\theta)}
          {(1 + \tan^2(\theta))^{\nicefrac{(u+v)}{2}}}
          \tan(\theta)\sec^2(\theta)
        }\diff{\theta}\\
      =\\
        & 2\left(\frac{v}{u}\right)^{\nicefrac{u}{2}}
        \int_0^{\nicefrac{\pi}{2}}{
          \dfrac{\tan^{u-1}(\theta)\sec^2(\theta)}{\sec^{u+v}(\theta)}
        }\diff{\theta}\\
      =\\
        & 2\left(\frac{v}{u}\right)^{\nicefrac{u}{2}}
        \int_0^{\nicefrac{\pi}{2}}{
          \sin^{u-1}(\theta)\cos^{v-2}\cos(\theta)
        }\diff{\theta}
    \end{longderivation}
    Para continuar, se realiza el cambio de variable
    \[\tilde{n}=\sin(\theta)\]
    Cuando $\theta=0$, $\tilde{n}=0$ y cuando $\theta=\nicefrac{\pi}{2}$, $\tilde{n}=1$.
    Por otra parte, dado que $\cos^2(\theta) + \sin^2(\theta) = 1$, entonces
    $\cos(\theta) = (1-\tilde{n}^2)^{\nicefrac{1}{2}}$. Recordando que
    $\odv*{\sin(\theta)}{\theta} = \cos(\theta)$, se obtiene la siguiente igualdad
    \begin{longderivation}
        & 2\left(\frac{v}{u}\right)^{\nicefrac{u}{2}}
        \int_0^{\nicefrac{\pi}{2}}{
          \sin^{u-1}(\theta)\cos^{v-2}\cos(\theta)
        }\diff{\theta}\\
      =\\
        & 2\left(\frac{v}{u}\right)^{\nicefrac{u}{2}}
        \int_0^1{
          \tilde{n}^{u-2}(1-\tilde{n}^2)^{(\nicefrac{v}{2})-1}
          \tilde{n}
        }\,\diff{\tilde{n}}
    \end{longderivation}
    Por último, ser realiza el cambio de variable
    \[t = \tilde{n}^2\]
    Los límites de integración se mantienen. Recordando que
    $\odv*{\tilde{n}^2}{\tilde{n}}=2\tilde{n}$,
    se obtienen las siguientes igualdades
    \begin{longderivation}
        & 2\left(\frac{v}{u}\right)^{\nicefrac{u}{2}}
          \int_0^1{
            \tilde{n}^{u-2}(1-\tilde{n}^2)^{(\nicefrac{v}{2})-1}
            \tilde{n}
          }\,\diff{\tilde{n}}\\
      =\\
        & 2\left(\frac{v}{u}\right)^{\nicefrac{u}{2}}
        \int_0^1{
          \frac{1}{2}t^{(\nicefrac{u}{2})-1}(1 - t)^{(\nicefrac{v}{2})-1}
        }\diff{t}\\
      =\\
        & \left(\frac{v}{u}\right)^{\nicefrac{u}{2}}
        B\left(\frac{u}{2},\frac{v}{2}\right)\\
      =\\
        & \left(\frac{u}{v}\right)^{-\nicefrac{u}{2}}
        B\left(\frac{u}{2},\frac{v}{2}\right)
    \end{longderivation}
    \item Denotando:
    \[
      A = \displaystyle\frac{\left(\frac{u}{v}\right)^{\nicefrac{u}{2}}}
      {B\left(\frac{u}{2}\text{,}\frac{v}{2}\right)}
    \]
    Que es el término constante de $f$. Manteniendo el cambio de 
    variable se obtiene las siguientes igualdades:

    \begin{longderivation}
      &\text{E}[F]\\
      =\\
      &A\int_{-\infty}^{\infty}xf(x)\diff{x}\\
      =\\
      &A\int_{0}^{\infty}x\frac{x^{(\nicefrac{u}{2})-1}}
      {\left(1+\frac{u}{v}x\right)^{\nicefrac{(u+v)}{2}}}\diff{x}\\
      =\\
      &A\int_{0}^{\infty}\frac{x^{\nicefrac{u}{2}}}
      {\left(1+\frac{u}{v}x\right)^{\nicefrac{(u+v)}{2}}}\\
      \why[=]{\text{Cambio de variable}}\\
      &A\int_{0}^{\frac{\pi}{2}}\frac{\left(\frac{v}{u}\tan^2(\theta)\right)}
      {(1+\tan^2(\theta))^{\nicefrac{(u+v)}{2}}}
      \frac{2v}{u}\tan(\theta)\sec^2(\theta)\diff{\theta}\\
      =\\
      &2\left(\frac{v}{u}\right)^{(\nicefrac{u}{2})+1}A
      \int_{0}^{\frac{\pi}{2}}\frac{\tan^{u+1}\theta}
      {(\sec^2(\theta))^{\nicefrac{(u+1)}{2}}}
      \sec^2(\theta)\diff{\theta}\\
      =\\
      &\left(\frac{v}{u}\right)^{(\nicefrac{u}{2})+1}A
      \int_{0}^{\frac{\pi}{2}}\tan^{u+1}(\theta)
      \sec^{2-u-v}(\theta)\diff{\theta}\\
      =\\
      &\left(\frac{v}{u}\right)^{(\nicefrac{u}{2})+1}A
      \int_{0}^{\frac{\pi}{2}}\sin^{u+1}(\theta)
      \cos^{v-3}(\theta)\diff{\theta}\\
    \end{longderivation}
    Haciendo la sustitución presentada anteriormente $t=\sin(\theta)$, 
    se continua como sigue:
    \begin{longderivation}
      &2A\left(\frac{v}{u}\right)^{(\nicefrac{u}{2})+1}
      \int_{0}^{1}t^{u+1}(1-t^2)^{\frac{(v-4)}{2}}\diff{t}\\
    \end{longderivation}
    Ahora, con la sustitución $w=t^2$. Cuando $t=0$, $w=0$, cuando 
    $t=1$, $w=1$. Además, $\diff{w}=2t\diff{t}$. 
    Luego, la integral se transforma en:
    \begin{longderivation}
      &A\left(\frac{v}{u}\right)^{(\nicefrac{u}{2})+1}
      \int_{0}^{1}w^{\nicefrac{u}{2}}(1-w)^{(\nicefrac{v}{2})-2}\diff{w}\\
      \why[=]{\text{Definición de $B$}}\\
      &A\left(\frac{v}{u}\right)^{(\nicefrac{u}{2})+1}
      B\left(\frac{u}{2}+1\text{,}\frac{v}{2}-1\right)\\
    \end{longderivation}
    Recordando que $B(x\text{,}y)=\displaystyle\frac{\Gamma(x)\Gamma(y)}
    {\Gamma(x+y)}$, y expandiendo $A$ se obtiene la expresión:
    \begin{longderivation}
      &\frac{\left(\frac{u}{v}\right)^{\nicefrac{u}{2}}}
      {B\left(\frac{u}{2}\text{,}\frac{v}{2}\right)}\left(\frac{v}{u}\right)^
      {(\nicefrac{u}{2})+1}
      \frac{\Gamma\left(\frac{u}{2}+1\right)\Gamma\left(\frac{v}{2}-1\right)}
      {\Gamma\left(\frac{u+v}{2}\right)}\\
      =\\
      &\left(\frac{v}{u}\right)\frac{\Gamma\left(\frac{u+v}{2}\right)}
      {\Gamma\left(\frac{u}{2}\right)\Gamma\left(\frac{v}{2}\right)}
      \frac{\Gamma\left(\frac{u}{2}+1\right)\Gamma\left(\frac{v}{2}-1\right)}
      {\Gamma\left(\frac{u+v}{2}\right)}\\
      =\\
      &\left(\frac{v}{u}\right)\frac{\frac{u}{2}\Gamma\left(\frac{u}{2}\right)
      \Gamma\left(\frac{v}{2}-1\right)}{\Gamma\left(\frac{u}{2}\right)
      \left(\frac{v}{2}-1\right)\Gamma\left(\frac{v}{2}-1\right)}\\
      =\\
      &\left(\frac{v}{u}\right)\left(\frac{u}{2}\right)
      \frac{1}{\left(\frac{v}{2}-1\right)}\\
      =\\
      &\left(\frac{v}{2}\right)\left(\frac{2}{v-2}\right)\\
      =\\
      &\frac{v}{v-2}
    \end{longderivation}
    Así, E$[F]=\displaystyle\frac{v}{v-2}$.
    \item Para demostrar la varianza, el procedimiento es casi identico
      al anterior. Se denota $A$ de la misma manera, y se empieza con la
      sustituión $\displaystyle\frac{u}{v}x=\tan^2\theta$.
      \begin{longderivation}
        &\text{Var}[F]\\
        =\\
        &\text{E}[F^2]-\text{E}^2[F]\\
        =\\
        &\int_{-\infty}^{\infty}x^2f(x)\diff{x}-\left(\frac{v}{v-2}\right)^2\\
        =\\
        &A\int_{0}^{\infty}x^2\frac{x^{\left(\nicefrac{u}{2}\right)-1}}
        {\left(1+\frac{u}{v}x\right)^{\nicefrac{(u+v)}{2}}}\diff{x}
        -\left(\frac{v}{v-2}\right)^2\\
        =\\
        &A\int_{0}^{\infty}\frac{x^{(\nicefrac{u}{2})+1}}
        {\left(1+\frac{u}{v}x\right)^{\nicefrac{(u+v)}{2}}}\diff{x}
        -\left(\frac{v}{v-2}\right)^2\\
        \why[=]{Haciendo el cambio de variable $\frac{u}{v}x=\tan^{2}(\theta)$}\\
        &A\int_{0}^{\frac{\pi}{2}}\frac{\left(\frac{v}{u}\right)^{(\nicefrac{u}{2})+1}
        (\tan^2(\theta))^{(\nicefrac{u}{2})+1}}
        {\left(1+\tan^2(\theta)\right)^{\nicefrac{(u+v)}{2}}}
        \frac{2v}{u}\tan(\theta)\sec^2(\theta)\diff{\theta}
        -\left(\frac{v}{v-2}\right)^2\\
        =\\
        &2A\left(\frac{v}{u}\right)^{(\nicefrac{u}{2})+2}
        \int_{0}^{\frac{\pi}{2}}
        \frac
        {\tan^{u+3}(\theta)}
        {\sec^{u+v-2}(\theta)}\diff{\theta}-\left(\frac{v}{v-2}\right)^2\\
        =\\
        &2A\left(\frac{v}{u}\right)^{(\nicefrac{u}{2})+2}
        \int_{0}^{\frac{\pi}{2}}
        \sin^{u+3}(\theta)\cos^{v-5}(\theta)\diff{\theta}
        -\left(\frac{v}{v-2}\right)^2\\
        \why[=]{Tomando la sustitución $t=\sin(\theta)$}\\
        &2A\left(\frac{v}{u}\right)^{(\nicefrac{u}{2})+2}
        \int_{0}^{1}t^{u+3}(1-t^2)^{\nicefrac{(v-6)}{2}}\diff{t}
        -\left(\frac{v}{v-2}\right)^2\\
        \why[=]{Por la sustitución $w=t^2$}\\
        &A\left(\frac{v}{u}\right)^{(\nicefrac{u}{2})+2}
        \int_{0}^{1}w^{(\nicefrac{u}{2})+1}(1-w)^{(\nicefrac{v}{2})-3}\diff{w}
        -\left(\frac{v}{v-2}\right)^2\\
        \why[=]{Definición de $B$ y expansión de $A$}\\
        &\left(\frac{v}{u}\right)^{(\nicefrac{u}{2})+2}
        \left(\frac{u}{v}\right)^{\nicefrac{u}{2}}
        \frac{\Gamma\left(\frac{u+v}{2}\right)}
        {\Gamma\left(\frac{u}{2}\right)\Gamma\left(\frac{v}{2}\right)}
        \frac{\Gamma\left(\frac{u}{2}+2\right)\Gamma\left(\frac{v}{2}-2\right)}
        {\Gamma\left(\frac{u+v}{2}\right)}
        -\left(\frac{v}{v-2}\right)^2\\
        \why[=]{Propiedades de $\Gamma$}\\
        &\left(\frac{v}{u}\right)^2
        \frac
        {\left(\frac{u}{2}+1\right)\left(\frac{u}{2}\right)
        \Gamma\left(\frac{u}{2}\right)\Gamma\left(\frac{v}{2}-2\right)}
        {\Gamma\left(\frac{u}{2}\right)\left(\frac{v}{2}-1\right)
        \left(\frac{v}{2}-2\right)\Gamma\left(\frac{v}{2}-2\right)}
        -\left(\frac{v}{v-2}\right)^2\\
        =\\
        &\left(\frac{v}{u}\right)^2\left(\frac{u}{2}\right)
        \left(\frac{u+2}{2}\right)
        \left(\frac{2}{v-2}\right)\left(\frac{2}{v-4}\right)
        -\left(\frac{v}{v-2}\right)^2\\
        =\\
        &\frac{v^2(u)(u+2)}{u^2(v-2)(v-4)}-\left(\frac{v}{v-2}\right)^2\\
        =\\
        &\frac{v^2(u+2)}{u(v-2)(v-4)}-\left(\frac{v}{v-2}\right)^2\\
        =\\
        &\frac{v^2(u+2)(v-2)-v^2u(v-4)}{u(v-2)^2(v-4)}\\
        =\\
        &\frac{v^2[(u+2)(v-2)-u(v-4)]}{u(v-2)^2(v-4)}\\
        =\\
        &\frac{v^2[uv-2u+2v-4-uv+4u]}{u(v-2)^2(v-4)}\\
        =\\
        &\frac{v^2[2u+2v-4]}{u(v-2)^2(v-4)}\\
        =\\
        &\frac{2v^2[u+v-2]}{u(v-2)^2(v-4)}
      \end{longderivation}
      Así Var$[F]=\displaystyle\frac{2v^2(u+v-2)}{u(v-2)^2(v-4)}$, 
      y se concluye la demotración
  \end{enumerate}
\end{Demo}


\subsubsection{Distribución Weibull}
\begin{Def}
  Sea $X$ una variable aleatoria continua. $X$ tiene distribución Weibull
  de parámetros $\lambda,\alpha\in\R^+$ cuando su función de densidad
  es
  \[
    f(x) = \lambda^\alpha\alpha x^{\alpha-1}e^{-(\lambda x)^\alpha}
    \qquad (x > 0)
  \]
\end{Def}
\begin{Teo}
  Sean $X$ una variable aleatoria continua con distribución Weibull de parámetros
  $\lambda,\alpha\in\R^+$ y $f$ su función de densidad. Entonces,
  \begin{enumerate}
    \item Para todo $x\in\R$, $f(x)\geq0$.
    \item $\int_{\R}f(x)\diff{x}=1$.
    \item $\text{E}[X]=\frac{1}{\lambda\alpha}
    \Gamma\left(\frac{1}{\alpha}\right)$.
    \item $\text{Var}[X]=\frac{1}{\lambda^2\alpha}\left(
      2\Gamma\left(\frac{2}{\alpha}\right)
      + \frac{1}{\alpha}\Gamma^2\left(\frac{1}{\alpha}\right)
    \right)$.
  \end{enumerate}
\end{Teo}
\begin{Demo}~
  \begin{enumerate}
    \item Dado que los términos en la definición de $f$ son no negativos,
    se concluye que, para todo $x\in\R$, $f(x)\geq0$.
    \item~
    \begin{longderivation}
        & \int_0^{\infty}\lambda^\alpha\alpha x^{\alpha-1}
        e^{-(\lambda x)^\alpha}\diff{x}\\
      \why[=]{Tomando $t=(\lambda x)^{\alpha}$}\\
        & \int_0^{\infty}e^{-t}\diff{t}\\
      =\\
        & 1
    \end{longderivation}
    \item~
    \begin{longderivation}
        & \text{E}[X]\\
      =\\
        & \int_0^{\infty}\lambda^\alpha\alpha x^{\alpha}
        e^{-(\lambda x)^\alpha}\diff{x}\\
      \why[=]{Tomando $t=(\lambda x)^{\alpha}$, $x=\frac{t^{1/\alpha}}{\lambda}$}\\
        & \frac{1}{\lambda}\int_0^{\infty}t^{1/\alpha}e^{-t}\diff{t}\\
      =\\
        & \frac{1}{\lambda}\Gamma\left(\frac{1}{\alpha}+1\right)\\
      =\\
        & \frac{1}{\lambda\alpha}\Gamma\left(\frac{1}{\alpha}\right)
    \end{longderivation}
    \item~
    \begin{longderivation}
        & \text{Var}[X]\\
      =\\
        & \text{E}[X^2] - \text{E}^2[X]\\
      =\\
        & \int_0^{\infty}\lambda^\alpha\alpha x^{\alpha+1}
        e^{-(\lambda x)^\alpha}\diff{x}
        - \frac{1}{\lambda^2\alpha^2}\Gamma^2\left(\frac{1}{\alpha}\right)\\
      \why[=]{Tomando $t=(\lambda x)^{\alpha}$, $x=\frac{t^{1/\alpha}}{\lambda}$}\\
        & \int_0^{\infty}\frac{t^{2/\alpha}}{\lambda^2}e^{-t}\diff{t}
        - \frac{1}{\lambda^2\alpha^2}\Gamma^2\left(\frac{1}{\alpha}\right)\\
      =\\
        & \frac{1}{\lambda^2}\left(
          \Gamma\left(1 + \frac{2}{\alpha}\right) -
          \frac{1}{\alpha^2}\Gamma^2\left(\frac{1}{\alpha}\right)
        \right)\\
      =\\
        & \frac{1}{\lambda^2\alpha}\left(
          2\Gamma\left(\frac{2}{\alpha}\right) -
          \frac{1}{\alpha}\Gamma^2\left(\frac{1}{\alpha}\right)
        \right)
    \end{longderivation}
  \end{enumerate}
\end{Demo}

\subsection{Teoremas de Aproximación}
Algunas de las funciones de distribución presentadas pueden involucrar
cálculos de alta complejidad para computar. Esto motivó la búsqueda
de aproximación entre funciones para los casos en los que
posiblemente sea más complejo usar una función que otra. En esta
sección, se presentarán algunos de los teoremas a cerca
de la aproximación de la probabilidad de una distribución mediante
otra.
\subsubsection{Hipergeométrica a Binomial}
Se puede ver una similitud entre la \hyperref[dist:binom]{distribución binomial}
y la \hyperref[dist:hip]{distribución hipergeométrica}, pues si en esta última, manteniendo
un tamaño de muestra ($n$) fijo, a medida que aumentan el total de objetos ($N$ y $K$) bajo
ciertas condiciones, los eventos que esta distribución describe tienen a ser independientes.
Esto lleva al siguiente teorema de aproximación.

\begin{Teo}
  Sea $X$ una variable aleatoria con distribución hipergeométrica de
  parámetros $N,K,n$. Si para $\epsilon_1,\epsilon_2,\epsilon_3\in\R^+$ , $n > 1$ , $x > 0$,
  se tiene que
  \begin{align*}
    \frac{x-1}{K}         &< \epsilon_1\\
    \frac{n-x-1}{N-K}     &< \epsilon_2\\
    \frac{n-1}{N - n + 1} &< \epsilon_3
  \end{align*}
  entonces,
  % Falta revisar cómo queda
  \[
    \left|\dfrac{Hg(N,K,n)(x)}{B\left(n,\frac{K}{N}\right)(x)} - 1\right| 
    < (\epsilon_1 + 1)^x(\epsilon_2+1)^{n-x}(\epsilon_3+1)^n - 1
  \]
  Nótese que la expresión en valor absoluto corresponde al error entre la
  función de masa de una distribución hipergeométrica y una binomial con ciertos
  parámetros.
\end{Teo}

Antes de comenzar con la demostración de este teorema, se presenta el siguiente lema, el
cual será de utilidad para obtener el resultado presentado.
\begin{Lema}
  Sean $r\in\Z^+$, $\left\{S_{k,n}\right\}_{1}^{r}$ una colección de $r$ sucesiones
  en función de $n$ las cuales convergen a $1$ y $\{\epsilon_k\}_1^r$ una colección
  de $r$ reales positivos. Si para un $N\in\N$, se tiene que
  \[1\leq k\leq r\quad\land\quad n\geq N \quad\To\quad |S_{k,n} - 1| \leq \epsilon_k\]
  entonces,
  \[n\geq N \To \left|\prod_{k=1}^r S_{k,n} -1\right| < \prod_{k=1}^r(\epsilon_k + 1) - 1\]
\end{Lema}
\begin{Demo}
  Supóngase la existencia de este $N$. Tomando una colección con $r+1$
  sucesiones con una colección respectiva de cotas $\{\epsilon_k\}_1^{r+1}$ para la diferencia
  de cada una con $1$, se tiene lo siguiente
  \begin{longderivation}
    & \left|\prod_{k=1}^{r+1}S_{k,n} - 1\right|\\
  =\\
    & \left|S_{r+1,n}\prod_{k=1}^{r}S_{k,n} - 1\right|\\
  =\\
    & \left|
      (S_{r+1,n} - 1)\left(\prod_{k=1}^{r}S_{k,n} - 1\right)
      + (S_{r+1,n} - 1)
      + \left(\prod_{k=1}^{r}S_{k,n} - 1\right)
    \right|\\
  \leq\\
    &\left|(S_{r+1,n} - 1)\left(\prod_{k=1}^{r}S_{k,n} - 1\right)\right|
    + \left|(S_{r+1,n} - 1)\right|
    + \left|\left(\prod_{k=1}^{r}S_{k,n} - 1\right)\right|\\
  <\\
    & \left|\prod_{k=1}^{r}S_{k,n} - 1\right|(\epsilon_{r+1}+1) + \epsilon_{r+1}
  \end{longderivation}

  Se define entonces la siguiente función recursiva
  \begin{align*}
    f(1) &= \epsilon_1\\
    f(n+1) &= f(n)(\epsilon_{n+1} + 1) + \epsilon_{n+1}
  \end{align*}

  Por el procedimiento anterior, es fácil ver que esta función cumple acotar la diferencia
  del producto de $n$ sucesiones y $1$ con las condiciones del enunciado. Se afirma que
  \[f(n) = \prod_{k=1}^n(\epsilon_k + 1) - 1\]
  Caso base: $n=1$, efectivamente $f(1) = \epsilon_1$. Para $n=2$, por definición
  \[f(2) = \epsilon_1(\epsilon_2+1)+\epsilon_2 = (\epsilon_1+1)(\epsilon_2+1)-1\]
  Paso inductivo: supóngase que, para algún $n\geq2$,
  \[f(n) = \prod_{k=1}^n(\epsilon_k + 1) - 1\]
  Entonces,
  \begin{longderivation}
      & f(n+1)\\
    =\\
      & f(n)(\epsilon_{n+1} + 1) + \epsilon_{n+1}\\
    =\\
      & \left(\prod_{k=1}^n(\epsilon_k + 1) - 1\right)(\epsilon_{n+1} + 1)
      + \epsilon_{n+1}\\
    =\\
      & \prod_{k=1}^{n+1}(\epsilon_k + 1) - \epsilon_{n+1} - 1 + \epsilon_{n+1}\\
    =\\
      & \prod_{k=1}^{n+1}(\epsilon_k + 1) - 1
  \end{longderivation}
  Con lo que
  \[n\geq N \To \left|\prod_{k=1}^r S_{k,n} -1\right| < \prod_{k=1}^r(\epsilon_k + 1) - 1\]
\end{Demo}
Siguiendo ahora con el teorema\dots
\begin{Demo}
  Inicialmente, se expresará la función de masa de $X$ en otros términos
  \begin{longderivation}
      & \dfrac{\binom{K}{x}\binom{N-K}{n-x}}{\binom{N}{n}}\\
    =\\
      & \frac{K!}{(K-x)!\,x!}\,\frac{(N-K)!}{(N-K-n+x)!\,(n-x)!}\,\frac{(N-n)!\,n!}{N!}\\
    =\\
      & \binom{n}{x}\dfrac{\prod_{i=1}^{K}i}{\prod_{i=1}^{K-x}i}
      \dfrac{\prod_{j=1}^{N-K}j}{\prod_{j=1}^{N-K-n+x}\mspace{-20mu}j\mspace{20mu}}
      \dfrac{\prod_{s=1}^{N-n}s}{\prod_{s=1}^{N}s}\\
    =\\
      & \binom{n}{x}\smashoperator[r]{\prod_{i=K-x+1}^{K}}i\mspace{20mu}
      \smashoperator[r]{\prod_{j=N-K-n+x+1}^{N-K}}j\mspace{35mu}
      \dfrac{1}{\smashoperator[r]{\prod_{s=N-n+1}^{N}}s\mspace{20mu}}\\
    =\\
      & \binom{n}{x}\prod_{i=0}^{x-1}(K-i)
      \smashoperator{\prod_{j=0}^{n-x-1}}(N-K-j)
      \dfrac{1}{\prod_{s=0}^{n-1}(N-s)}\\
    =\\
      &\binom{n}{x}\left(\frac{K}{N}\right)^x\left(\frac{N-K}{N}\right)^{n-x}
      \dfrac{\prod_{i=0}^{x-1}(K-i)}{K^x}
      \dfrac{\smashoperator[r]{\prod_{j=0}^{n-x-1}}(N-K-j)}{(N-K)^{n-x}}
      \dfrac{N^n}{\prod_{s=0}^{n-1}(N-s)}\\
    =\\
      & B\left(n,\frac{K}{N}\right)(x)\,\prod_{i=0}^{x-1}\left(1-\frac{i}{K}\right)
      \prod_{j=0}^{n-x-1}\left(1 - \frac{j}{N-K}\right)
      \prod_{s=0}^{n-1}\left(1 + \frac{s}{N-s}\right)
  \end{longderivation}

  En este proceso no se toma en cuenta el caso en el que $x=0$ o $x=n$. Estos casos se
  resolverán posterior a tratar con la última expresión.

  Tomando en cuenta este resultado,
  \begin{longderivation}<1.5>
      & \left|\dfrac{Hg(N,K,n)(x)}{B\left(n,\frac{K}{N}\right)(x)} - 1\right|\\
    =\\
      & \left|\prod_{i=0}^{x-1}\left(1-\frac{i}{K}\right)
      \prod_{j=0}^{n-x-1}\left(1 - \frac{j}{N-K}\right)
      \prod_{s=0}^{n-1}\left(1 + \frac{s}{N-s}\right) - 1\right|
  \end{longderivation}

  Nótese que cada término en cada productorio tiende a $1$ cuando $K,N,N-K$
  tienden a infinito. Con esto basta para demostrar la convergencia, debido a que
  estas condiciones de tendencia para $N,K$ y $N-K$ se deben a que
  $\frac{K}{N}$ debe ser un número entre $0$ y $1$.

  En los productorios, se ven involucradas sucesiones las cuales convergen
  a $0$ y además, son sencillas de acotar. Entonces, como
  \begin{align*}
    \left|1 - \frac{i}{K} - 1\right| &= \frac{i}{K} \leq \frac{x-1}{K}\\[10pt]
    \left|1 - \frac{j}{N-K} - 1\right| &= \frac{j}{N-K} \leq \frac{n-x-1}{N-K}\\[10pt]
    \left|1 + \frac{s}{N-s} - 1\right| &= \frac{s}{N-s} \leq \frac{n-1}{N-n+1}
  \end{align*}
  
  Dado que las expresiones a la derecha de cada desigualdad representan sucesiones
  decrecientes en función de $N$, $K$ y $N-K$ respectivamente, se tiene que, si para
  valores de estas variables, se toman $\epsilon_1, \epsilon_2, \epsilon_3\in\R^+$
  tales que
  \begin{align*}
    \frac{x-1}{K}     &< \epsilon_1\\
    \frac{n-x-1}{N-K} &< \epsilon_2\\
    \frac{n-1}{N-n+1} &< \epsilon_3
  \end{align*}
  Entonces, por el lema,
  \begin{align*}
    \left|\prod_{i=0}^{x-1}\left(1-\frac{i}{K}\right) - 1\right| 
    &< (\epsilon_1 + 1)^x - 1\\[10pt]
    \left|\prod_{j=0}^{n-x-1}\left(1 - \frac{j}{N-K}\right) - 1\right|
    &< (\epsilon_2 + 1)^{n-x} - 1\\[10pt]
    \left|\prod_{s=0}^{n-1}\left(1 + \frac{s}{N-s}\right) - 1\right|
    &< (\epsilon_3 + 1)^n - 1
  \end{align*}

  Denotando cada uno de estos productos como $P_1$, $P_2$ y $P_3$ respectivamente,
  aplicando nuevamente el lema, se obtiene que
  \[
    |P_1\,P_2\,P_3 - 1|<
    (\epsilon_1+1)^x(\epsilon_2+1)^{n-x}(\epsilon_3+1)^n - 1
  \]
  Recordando que todo lo anterior se hizo bajo la suposición de que $x > 0$ y
  $ x \not= n$. Para $x = 0$
  \begin{longderivation}
      &\left|
        \dfrac{Hg(N,K,n)(0)}{B\left(n,\frac{K}{N}\right)(0)}-1
      \right|\\
    =\\
      &\left|
        \dfrac{\binom{N-K}{n}}{\binom{N}{n}}
        \left(\frac{N-K}{N}\right)^{-n}
        -1
      \right|\\
    =\\
      & \left|
        \prod_{j=0}^{n-1}\left(1 - \frac{j}{N-K}\right)
        \prod_{s=0}^{n-1}\left(1 + \frac{s}{N-s}\right) - 1
      \right|\\
    \why[<]{Aplicando el lema únicamente para las suceciones en estos productorios}\\
      & (\epsilon_2+1)^n(\epsilon_3+1)^n-1
  \end{longderivation}
  Para $x=n$
  \begin{longderivation}
      &\left|
        \dfrac{Hg(N,K,n)(n)}{B\left(n,\frac{K}{N}\right)(n)}-1
      \right|\\
    =\\
      &\left|
        \dfrac{\binom{K}{n}}{\binom{N}{n}}\left(\frac{K}{N}\right)^{-n}
        -1
      \right|\\
    =\\
      &\left|
        \prod_{i=0}^{n-1}\left(1 - \frac{i}{K}\right)
        \prod_{s=0}^{n-1}\left(1 + \frac{s}{N-s}\right) - 1
      \right|\\
    \why[<]{Aplicando el lema únicamente para las sucesiones en estos productos}\\
      & (\epsilon_1+1)^n(\epsilon_3+1)^n-1
  \end{longderivation}
  Para $n=1$ o $n=0$, el error es nulo.
\end{Demo}
\subsubsection{Teorema Central del Límite}
Es de interés conocer la distribución de una variable aleatoria la
cual se puede escribir como combinación de varias variables
aleatorias. Estas combinaciones son llamadas \emph{estadísticos}.
Un ejemplo puede ser simplemente la suma o producto de dos variables
aleatorias. Uno de los más simples pero a su vez el que llevó
a uno de los resultados más importantes es el de la media.
\begin{Teo}
  Sea $\{X_n\}_{n\in\Z^+}$ una colección de variables aleatorias
  independientes e igualmente distribuidas, donde $\mu$ es su media y $\sigma^2$
  su varianza. denotando $\overline{X_n}=\frac{1}{n}\sum_{k=1}^nX_k$, Entonces,
  \[\sqrt{n}\,\overline{X}_n\longrightarrow N(\mu,\sigma^2)\]
  en distribución.
\end{Teo}

Antes de continuar con la demostración, se presenta un resultado sobre suceciones.
\begin{Lema}
  Sean $x_n,y_n$ sucesiones complejas y $M\in\R^+$ tal que, para todo
  $n\in\Z^+$,
  \begin{align*}
    |x_n| &\leq M\\
    |y_n| &\leq M
  \end{align*}
  Entonces, para $n\in\Z^+$,
  \[
    \left|\prod_{k=1}^n x_k - \prod_{j=1}^n y_j\right| \leq 
    M^{n-1}\sum_{k=1}^n |x_k - y_k|
  \]
\end{Lema}
\begin{Demo}
  Sean $x_n$ y $y_n$ sucesiones con las condiciones enunciadas. La
  demostración se hará por inducción.
  
  Caso base ($n=1$):
  \[|x_1-y_1| \leq |x_1-y_1|\]

  Paso inductivo:\\
  Supóngase que la propiedad se mantiene para algún $n\in\Z^+$.
  \begin{longderivation}
      & \left|\prod_{k=1}^{n+1} x_k - \prod_{j=1}^{n+1} y_j\right|\\
    =\\
      & \frac{1}{2}\left|
        (x_{n+1}-y_{n+1})\left(
          \prod_{k=1}^n x_k + \prod_{j=1}^n y_j
        \right)
        + (x_{n+1}+y_{n+1})\left(
          \prod_{k=1}^n x_k - \prod_{j=1}^n y_j
        \right)
      \right|\\
    \leq\\
      & \frac{1}{2}|x_{n+1}-y_{n+1}|\left(
        \prod_{k=1}^n |x_k| - \prod_{j=1}^n |y_j|
      \right)
      + \frac{1}{2}(|x_{n+1} + y_{n+1})\left|
        \prod_{k=1}^n x_k - \prod_{j=1}^n y_j
      \right|\\
    \leq\\
      & \frac{1}{2}|x_{n+1}-y_{n+1}|\left(
        2M^n
      \right)
      + \frac{1}{2}(2M)
        M^{n-1}\sum_{k=1}^n |x_k - y_k|\\
    =\\
      & M^n\sum_{k=1}^{n+1} |x_k - y_k|
  \end{longderivation}
\end{Demo}

Procediendo con la demostración del teorema.
\begin{Demo}
  Se define $Z_n = \frac{X_n-\mu}{\sigma}$. Para todo $n\in\Z^+$,
  \begin{align*}
    \text{E}[Z_n]&= \text{E}\left[\frac{X_n-\mu}{\sigma}\right]
    = 0\\
    \text{Var}[Z_n] &= \text{E}\left[\frac{X_n-\mu}{\sigma}\right]
    = 1
  \end{align*}
  El enunciado del teorema resulta equivalente a mostrar que
  \[\sqrt{n}\,\overline{Z}_n\longrightarrow N(0,1)\]

  Sea $\phi$ la función característica de $Z_n$ con $n\in\Z^+$.
  Se mostrará una convergencia puntual de la función
  característica. Es decir,
  \[\lim_{n\to\infty}\phi_{\sqrt{n}\overline{Z}_n}(t) = e^{-t^2/2}\]
  \begin{longderivation}
      & \phi_{\sqrt{n}\overline{Z}_n}(t)\\
    =\\
      & \text{E}\left[
        \exp\left(it\frac{Z_1 + Z_2 +\dots+Z_n}{\sqrt{n}}\right)
      \right]\\
    =\\
      & \text{E}\left[
        \exp\left(
          \prod_{k=1}^n\exp\left(i\frac{t}{\sqrt{n}}Z_k\right)
        \right)
      \right]\\
    \why[=]{
      Dado que las variables aleatorias en $\{Z_k\}_{k\in\Z^+}$ son
      independientes entre si
    }\\
      & \prod_{k=1}^n\text{E}\left[
        \exp\left(i\frac{t}{\sqrt{n}}Z_k\right)
      \right]\\
    =\\
      & \prod_{k=1}^n\phi\left(\frac{t}{\sqrt{n}}\right)\\
    =\\
      & \left[\phi\left(\frac{t}{\sqrt{n}}\right)\right]^n\\
  \end{longderivation}

Dado que el valor esperado y la varianza de las variables $Z_n$
existen, se tiene por las
\hyperref[Teo:prop_carac]{propiedades de la función característica}
que $\phi(0)=1$, $\phi'(0)=0$ y $\phi''(0) = -1$.
Por el teorema de Taylor, se tiene que para algún $\xi\in\R$
\[\phi\left(\frac{t}{\sqrt{n}}\right) = 
1 -\frac{t^2}{2n} + \frac{\phi'''(\xi)}{3!}\left(\frac{t}{\sqrt{n}}\right)^3\]

Tomando $f_n(t) = \frac{t^2}{2} + \frac{\phi'''(\xi)}{3!}\frac{t^3}{\sqrt{n}}$,
hace falta mostrar que, puntualmente,
\[\left(1 - \frac{f_n(t)}{n}\right)^n \longrightarrow e^{-\nicefrac{t^2}{2}}\]

Nótese que, puntualmente, $f_n$ converge a $\frac{t^2}{2}$ y es acotada.
Sean $t\in\R$ y $\epsilon>0$, entonces, por lo mencionado anteriormente, existe
$N\in\Z^+$ tal que para todo $n\geq N$,
\begin{align*}
  \left|f_n(t) - \frac{t^2}{2}\right| &< \epsilon\\
  \left|1 - \frac{f_n(t)}{n}\right|
    &\leq 1 + \frac{\underset{n\in\Z^+}{\sup} |f_n(t)|}{n}\\
  \left|
    \left(1 - \frac{t^2}{2n}\right)^n - e^{-\nicefrac{t^2}{2}}
  \right| &< \epsilon
\end{align*}
Entonces,
\begin{longderivation}
    & \left|
      \left(1 - \frac{f_n(t)}{n}\right)^n - e^{-\nicefrac{t^2}{2}}
    \right|\\
  =\\
    & \left|
      \left(1 - \frac{f_n(t)}{n}\right)^n - \left(1 - \frac{t^2}{2n}\right)^n
      + \left(1 - \frac{t^2}{2n}\right)^n - e^{-\nicefrac{t^2}{2}}
    \right|\\
  \leq\\
    & \left|
      \left(1 - \frac{f_n(t)}{n}\right)^n - \left(1 - \frac{t^2}{2n}\right)^n
    \right| + \left|
      \left(1 - \frac{t^2}{2n}\right)^n - e^{-\nicefrac{t^2}{2}}
    \right|\\
  \why[<]{Usando el lema presentado anteriormente}\\
    & \left(1 + \frac{\underset{n\in\Z^+}{\sup} |f_n(t)|}{n}\right)^n
    \sum_{k=1}^{n-1}\frac{|f_n(t)-\nicefrac{t^2}{2}|}{n}
    + \epsilon\\
  <\\
    & \left(1 + \frac{\underset{n\in\Z^+}{\sup} |f_n(t)|}{n}\right)^n
    \left(1 + \frac{\underset{n\in\Z^+}{\sup} |f_n(t)|}{n}\right)^{-1}
  \epsilon + \epsilon\\
\end{longderivation}
En el sumando izquierdo, fuera del factor $\epsilon$, hay dos sucesiones
numéricas, una de estas es creciente y converge a una exponencial.
La otra es decreciente y converge a $1$. Es decir, ambos valores son finitos.
Así, se tiene que puntualmente
\[\phi_{\sqrt{n}\overline{Z}_n}(t)\longrightarrow e^{-\nicefrac{t^2}{2}}\]
Por unicidad de la función característica, se concluye la demostración.
\end{Demo}

\nocite{LilianaBlanco}
\printbibliography
\end{document}