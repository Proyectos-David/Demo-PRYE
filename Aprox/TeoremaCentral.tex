\subsubsection{Teorema Central del Límite}
Es de interés conocer la distribución de una variable aleatoria la
cual se puede escribir como combinación de varias variables
aleatorias. Estas combinaciones son llamadas \emph{estadísticos}.
Un ejemplo puede ser simplemente la suma o producto de dos variables
aleatorias. Uno de los más simples pero a su vez el que llevó
a uno de los resultados más importantes es el de la media.
\begin{Teo}
  Sea $\{X_n\}_{n\in\Z^+}$ una colección de variables aleatorias
  independientes e igualmente distribuidas, donde $\mu$ es su media y $\sigma^2$
  su varianza. denotando $\overline{X_n}=\frac{1}{n}\sum_{k=1}^nX_k$, Entonces,
  \[\sqrt{n}\,\overline{X}_n\longrightarrow N(\mu,\sigma^2)\]
  en distribución.
\end{Teo}

Antes de continuar con la demostración, se presenta un resultado sobre suceciones.
\begin{Lema}
  Sean $x_n,y_n$ sucesiones complejas y $M\in\R^+$ tal que, para todo
  $n\in\Z^+$,
  \begin{align*}
    |x_n| &\leq M\\
    |y_n| &\leq M
  \end{align*}
  Entonces, para $n\in\Z^+$,
  \[
    \left|\prod_{k=1}^n x_k - \prod_{j=1}^n y_j\right| \leq 
    M^{n-1}\sum_{k=1}^n |x_k - y_k|
  \]
\end{Lema}
\begin{Demo}
  Sean $x_n$ y $y_n$ sucesiones con las condiciones enunciadas. La
  demostración se hará por inducción.
  
  Caso base ($n=1$):
  \[|x_1-y_1| \leq |x_1-y_1|\]

  Paso inductivo:\\
  Supóngase que la propiedad se mantiene para algún $n\in\Z^+$.
  \begin{longderivation}
      & \left|\prod_{k=1}^{n+1} x_k - \prod_{j=1}^{n+1} y_j\right|\\
    =\\
      & \frac{1}{2}\left|
        (x_{n+1}-y_{n+1})\left(
          \prod_{k=1}^n x_k + \prod_{j=1}^n y_j
        \right)
        + (x_{n+1}+y_{n+1})\left(
          \prod_{k=1}^n x_k - \prod_{j=1}^n y_j
        \right)
      \right|\\
    \leq\\
      & \frac{1}{2}|x_{n+1}-y_{n+1}|\left(
        \prod_{k=1}^n |x_k| - \prod_{j=1}^n |y_j|
      \right)
      + \frac{1}{2}(|x_{n+1} + y_{n+1})\left|
        \prod_{k=1}^n x_k - \prod_{j=1}^n y_j
      \right|\\
    \leq\\
      & \frac{1}{2}|x_{n+1}-y_{n+1}|\left(
        2M^n
      \right)
      + \frac{1}{2}(2M)
        M^{n-1}\sum_{k=1}^n |x_k - y_k|\\
    =\\
      & M^n\sum_{k=1}^{n+1} |x_k - y_k|
  \end{longderivation}
\end{Demo}

Procediendo con la demostración del teorema.
\begin{Demo}
  Se define $Z_n = \frac{X_n-\mu}{\sigma}$. Para todo $n\in\Z^+$,
  \begin{align*}
    \text{E}[Z_n]&= \text{E}\left[\frac{X_n-\mu}{\sigma}\right]
    = 0\\
    \text{Var}[Z_n] &= \text{E}\left[\frac{X_n-\mu}{\sigma}\right]
    = 1
  \end{align*}
  El enunciado del teorema resulta equivalente a mostrar que
  \[\sqrt{n}\,\overline{Z}_n\longrightarrow N(0,1)\]

  Sea $\phi$ la función característica de $Z_n$ con $n\in\Z^+$.
  Se mostrará una convergencia puntual de la función
  característica. Es decir,
  \[\lim_{n\to\infty}\phi_{\sqrt{n}\overline{Z}_n}(t) = e^{-t^2/2}\]
  \begin{longderivation}
      & \phi_{\sqrt{n}\overline{Z}_n}(t)\\
    =\\
      & \text{E}\left[
        \exp\left(it\frac{Z_1 + Z_2 +\dots+Z_n}{\sqrt{n}}\right)
      \right]\\
    =\\
      & \text{E}\left[
        \exp\left(
          \prod_{k=1}^n\exp\left(i\frac{t}{\sqrt{n}}Z_k\right)
        \right)
      \right]\\
    \why[=]{
      Dado que las variables aleatorias en $\{Z_k\}_{k\in\Z^+}$ son
      independientes entre si
    }\\
      & \prod_{k=1}^n\text{E}\left[
        \exp\left(i\frac{t}{\sqrt{n}}Z_k\right)
      \right]\\
    =\\
      & \prod_{k=1}^n\phi\left(\frac{t}{\sqrt{n}}\right)\\
    =\\
      & \left[\phi\left(\frac{t}{\sqrt{n}}\right)\right]^n\\
  \end{longderivation}

Dado que el valor esperado y la varianza de las variables $Z_n$
existen, se tiene por las
\hyperref[Teo:prop_carac]{propiedades de la función característica}
que $\phi(0)=1$, $\phi'(0)=0$ y $\phi''(0) = -1$.
Por el teorema de Taylor, se tiene que para algún $\xi\in\R$
\[\phi\left(\frac{t}{\sqrt{n}}\right) = 
1 -\frac{t^2}{2n} + \frac{\phi'''(\xi)}{3!}\left(\frac{t}{\sqrt{n}}\right)^3\]

Tomando $f_n(t) = \frac{t^2}{2} + \frac{\phi'''(\xi)}{3!}\frac{t^3}{\sqrt{n}}$,
hace falta mostrar que, puntualmente,
\[\left(1 - \frac{f_n(t)}{n}\right)^n \longrightarrow e^{-\nicefrac{t^2}{2}}\]

Nótese que, puntualmente, $f_n$ converge a $\frac{t^2}{2}$ y es acotada.
Sean $t\in\R$ y $\epsilon>0$, entonces, por lo mencionado anteriormente, existe
$N\in\Z^+$ tal que para todo $n\geq N$,
\begin{align*}
  \left|f_n(t) - \frac{t^2}{2}\right| &< \epsilon\\
  \left|1 - \frac{f_n(t)}{n}\right|
    &\leq 1 + \frac{\underset{n\in\Z^+}{\sup} |f_n(t)|}{n}\\
  \left|
    \left(1 - \frac{t^2}{2n}\right)^n - e^{-\nicefrac{t^2}{2}}
  \right| &< \epsilon
\end{align*}
Entonces,
\begin{longderivation}
    & \left|
      \left(1 - \frac{f_n(t)}{n}\right)^n - e^{-\nicefrac{t^2}{2}}
    \right|\\
  =\\
    & \left|
      \left(1 - \frac{f_n(t)}{n}\right)^n - \left(1 - \frac{t^2}{2n}\right)^n
      + \left(1 - \frac{t^2}{2n}\right)^n - e^{-\nicefrac{t^2}{2}}
    \right|\\
  \leq\\
    & \left|
      \left(1 - \frac{f_n(t)}{n}\right)^n - \left(1 - \frac{t^2}{2n}\right)^n
    \right| + \left|
      \left(1 - \frac{t^2}{2n}\right)^n - e^{-\nicefrac{t^2}{2}}
    \right|\\
  \why[<]{Usando el lema presentado anteriormente}\\
    & \left(1 + \frac{\underset{n\in\Z^+}{\sup} |f_n(t)|}{n}\right)^n
    \sum_{k=1}^{n-1}\frac{|f_n(t)-\nicefrac{t^2}{2}|}{n}
    + \epsilon\\
  <\\
    & \left(1 + \frac{\underset{n\in\Z^+}{\sup} |f_n(t)|}{n}\right)^n
    \left(1 + \frac{\underset{n\in\Z^+}{\sup} |f_n(t)|}{n}\right)^{-1}
  \epsilon + \epsilon\\
\end{longderivation}
En el sumando izquierdo, fuera del factor $\epsilon$, hay dos sucesiones
numéricas, una de estas es creciente y converge a una exponencial.
La otra es decreciente y converge a $1$. Es decir, ambos valores son finitos.
Así, se tiene que puntualmente
\[\phi_{\sqrt{n}\overline{Z}_n}(t)\longrightarrow e^{-\nicefrac{t^2}{2}}\]
Por unicidad de la función característica, se concluye la demostración.
\end{Demo}