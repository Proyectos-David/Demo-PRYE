\subsubsection{Distribución Weibull}
\begin{Def}
  Sea $X$ una variable aleatoria continua. $X$ tiene distribución Weibull
  de parámetros $\lambda,\alpha\in\R^+$ cuando su función de densidad
  es
  \[
    f(x) = \lambda^\alpha\alpha x^{\alpha-1}e^{-(\lambda x)^\alpha}
    \qquad (x > 0)
  \]
\end{Def}
\begin{Teo}
  Sean $X$ una variable aleatoria continua con distribución Weibull de parámetros
  $\lambda,\alpha\in\R^+$ y $f$ su función de densidad. Entonces,
  \begin{enumerate}
    \item Para todo $x\in\R$, $f(x)\geq0$.
    \item $\int_{\R}f(x)\diff{x}=1$.
    \item $\text{E}[X]=\frac{1}{\lambda\alpha}
    \Gamma\left(\frac{1}{\alpha}\right)$.
    \item $\text{Var}[X]=\frac{1}{\lambda^2\alpha}\left(
      2\Gamma\left(\frac{2}{\alpha}\right)
      + \frac{1}{\alpha}\Gamma^2\left(\frac{1}{\alpha}\right)
    \right)$.
  \end{enumerate}
\end{Teo}
\begin{Demo}~
  \begin{enumerate}
    \item Dado que los términos en la definición de $f$ son no negativos,
    se concluye que, para todo $x\in\R$, $f(x)\geq0$.
    \item~
    \begin{longderivation}
        & \int_0^{\infty}\lambda^\alpha\alpha x^{\alpha-1}
        e^{-(\lambda x)^\alpha}\diff{x}\\
      \why[=]{Tomando $t=(\lambda x)^{\alpha}$}\\
        & \int_0^{\infty}e^{-t}\diff{t}\\
      =\\
        & 1
    \end{longderivation}
    \item~
    \begin{longderivation}
        & \text{E}[X]\\
      =\\
        & \int_0^{\infty}\lambda^\alpha\alpha x^{\alpha}
        e^{-(\lambda x)^\alpha}\diff{x}\\
      \why[=]{Tomando $t=(\lambda x)^{\alpha}$, $x=\frac{t^{1/\alpha}}{\lambda}$}\\
        & \frac{1}{\lambda}\int_0^{\infty}t^{1/\alpha}e^{-t}\diff{t}\\
      =\\
        & \frac{1}{\lambda}\Gamma\left(\frac{1}{\alpha}+1\right)\\
      =\\
        & \frac{1}{\lambda\alpha}\Gamma\left(\frac{1}{\alpha}\right)
    \end{longderivation}
    \item~
    \begin{longderivation}
        & \text{Var}[X]\\
      =\\
        & \text{E}[X^2] - \text{E}^2[X]\\
      =\\
        & \int_0^{\infty}\lambda^\alpha\alpha x^{\alpha+1}
        e^{-(\lambda x)^\alpha}\diff{x}
        - \frac{1}{\lambda^2\alpha^2}\Gamma^2\left(\frac{1}{\alpha}\right)\\
      \why[=]{Tomando $t=(\lambda x)^{\alpha}$, $x=\frac{t^{1/\alpha}}{\lambda}$}\\
        & \int_0^{\infty}\frac{t^{2/\alpha}}{\lambda^2}e^{-t}\diff{t}
        - \frac{1}{\lambda^2\alpha^2}\Gamma^2\left(\frac{1}{\alpha}\right)\\
      =\\
        & \frac{1}{\lambda^2}\left(
          \Gamma\left(1 + \frac{2}{\alpha}\right) -
          \frac{1}{\alpha^2}\Gamma^2\left(\frac{1}{\alpha}\right)
        \right)\\
      =\\
        & \frac{1}{\lambda^2\alpha}\left(
          2\Gamma\left(\frac{2}{\alpha}\right) -
          \frac{1}{\alpha}\Gamma^2\left(\frac{1}{\alpha}\right)
        \right)
    \end{longderivation}
  \end{enumerate}
\end{Demo}