\subsubsection{Poisson}
Recordando que $e^x=\sum_{n=0}^\infty\frac{x^n}{n!}$, y que esta es una función
creciente cuyo valor es estrictamente positivo, se genera una función la cual
cumple la definición de función de masa, obteniendo la siguiente distribución.
\begin{Def}
  Sea $X$ una variable aleatoria discreta. $X$ sigue una distribución
  de Poisson con parámetro $\lambda\in\R^+$, denotada por $\text{Pois}(\lambda)$,
  cuando su función de masa es
  \[f(x) = P(X=x) = \frac{\lambda^xe^{-\lambda}}{x!} \qquad (x\in\N)\]
\end{Def}
\begin{Teo}
  Sea $X\sim\text{Pois}(\lambda)$. Entonces,
  \begin{enumerate}
    \item Para todo $x\in\Z$, $0\leq P(X=x)\leq1$.
    \item $\sum_{x\in\Z}P(X=x)=1$.
    \item $\text{E}[X] = \lambda$.
    \item $\text{Var}[X]=\lambda$.
  \end{enumerate}
\end{Teo}
\begin{Demo}~
  \begin{enumerate}
    \item Sea $x\in\N$. Entonces
    \[P(X=x) = \frac{\lambda^xe^{-\lambda}}{x!}\]
    Como los términos involucrados son no negativos, se concluye que $P(X=x)\geq0$.
    Para la otra parte de la desigualdad, dado que, para todo $\lambda\in\R$,
    \[e^{\lambda} = \sum_{n=0}^{\infty}\frac{\lambda^n}{n!}\]
    y la serie presentada es de términos no negativos para $\lambda\in\R^+$, se sigue
    que, para todo $x\in\N$,
    \[\frac{\lambda^x}{x!} \leq e^{\lambda}\]
    Así, para todo $x\in\N$,
    \[0\leq P(X=x)\leq1\]
    \item~
    \begin{longderivation}<1>
        & \sum_{x=0}^\infty\frac{\lambda^xe^{-\lambda}}{x!}\\
      =\\
        & e^{-\lambda}\sum_{x=0}^\infty\frac{\lambda^x}{x!}\\
      =\\
        & e^{-\lambda}\,e^\lambda\\
      =\\
        & 1
    \end{longderivation}
    \item~
    \begin{longderivation}<1>
        & \sum_{n=0}^\infty x\frac{\lambda^xe^{-\lambda}}{x!}\\
      =\\
        & \sum_{n=1}^\infty\frac{\lambda^xe^{-\lambda}}{(x-1)!}\\
      =\\
        & \lambda e^{-\lambda}\sum_{x=0}^\infty\frac{\lambda^x}{x!}\\
      =\\
        & \lambda
    \end{longderivation}
    \item~
    \begin{longderivation}<1>
        & \sum_{x=0}^\infty x^2\frac{\lambda^xe^{-\lambda}}{x!} - \lambda^2\\
      =\\
        & \sum_{n=1}^\infty x\frac{\lambda^xe^{-\lambda}}{(x-1)!} - \lambda^2\\
      =\\
        & \lambda\sum_{n=0}^\infty x\frac{\lambda^xe^{-\lambda}}{x!}
        + \lambda\sum_{n=0}^\infty \frac{\lambda^xe^{-\lambda}}{x!}
        - \lambda^2\\
      =\\
        & \lambda^2 + \lambda - \lambda^2\\
      =\\
        & \lambda
    \end{longderivation}
  \end{enumerate}

  Para finalizar, se mostrará la validez de los procedimientos usados para
  estos cálculos. Se afirma que la serie $\sum_x \left|x^t \frac{\lambda^x}{x!}\right|$
  converge para todo $t\in\R$. 
  Por el criterio de la razón, cuando $x\to\infty$,
  \[
    \left|(x+1)^t\frac{\lambda^{x+1}e^\lambda}{(x+1)!}\,
    \frac{x!}{x^t\lambda^x}\right|
    =
    \left(1 + \frac{1}{x}\right)^t\,\frac{\lambda}{x+1}
    \to 0 < 1
  \]
  La convergencia absoluta de la serie permite el reordenamiento de la misma y
  la separación en sumas.
\end{Demo}