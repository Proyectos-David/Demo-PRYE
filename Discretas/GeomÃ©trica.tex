\subsubsection{Distribucion geométrica}
\begin{Def}
    Sea $X$ una variable aleatoria discreta, $X$ sigue una distribución
    geométrica con parámetro $0<p<1$, denotada por $\text{Geo}(p)$, cuando su
    función de masa es
    \[
        f(x) = \text{P}(X=x)=p(1-p)^{x-1} \qquad (x \in \Z^+)
    \]
\end{Def}
\begin{Teo}
    Sea $X\sim\text{Geo}(p)$. Entonces,
    \begin{enumerate}
        \item Para todo $X\in\Z$, $0\leq P(X=x)\leq1$
        \item $\sum_{x\in\Z}P(X=x)=1$
        \item $\text{E}[X]=\frac{1}{p}$
        \item $\text{Var}[X]=\frac{1-p}{p^2}$
    \end{enumerate}
\end{Teo}
\begin{Demo}~
    \begin{enumerate}
        \item Dado que $0<p< 1$, entones $-1<p-1<0$, y así,
        para $x\in\Z^+$, $0<(p-1)^{x-1}<1$.
        Por otro lado multiplicando las desigualdades mienbro a miembro
        se obtien que $0<p(p-1)^{x-1}<1$. 
        \item \[
        \sum_{x\in\Z}P(X=x)
        \]
        Como $P(X\leq 0)=0$, entonces la suma resulta en
        \[
        \sum_{x=1 }^{\infty}p(1-p)^{x-1}
        \] 
        Como $p$ es constante, la suma es la serie geométrica de $1-p$, 
        luego, la expresión es igual a 
        \[
        p\sum_{x=1}^{\infty}(1-p)^{x-1} = p\left(\frac{1}{1-(1-p)}\right)=1
        \]
        \item Sea $f(p)=\sum_{x=0}^{\infty}(1-p)^{x}\quad(0<p<1)$.
        Dado que $0<p<1$, la función está bien definida en el dominio presentado.
        Al ser $f$ una representación en serie, se tiene que
        \[f(1-p)=\frac{1}{1-(1-p)}=\frac{1}{p}\]
        Así
        \begin{longderivation}
            & \odv*{f(1-p)}{p} = \odv*{\left(\frac{1}{p}\right)}{p}\\
            \iff\\
            &\odv*{\sum_{x=0}^{\infty}(1-p)^{x}}{p}=\odv*{\left(\frac{1}{p}\right)}{p}\\
            \iff\\
            &-\sum_{x=1}^{\infty}x(1-p)^{x-1}=-\frac{1}{p^2}\\
            \iff\\
            &\sum_{x=1}^{\infty}x(1-p)^{x-1}=\frac{1}{p^2}
        \end{longderivation}
        Multiplicando ambos términos por $p$, se obtiene la expresión
        \[
            \sum_{x=1}^{\infty}xp(1-p)^{x-1}=\frac{1}{p}
        \]
        El miembro de la derecha no es otra cosa que el valor 
        esperado de la distribución geométrica. Así pues 
        $\text{E}[X]=\dfrac{1}{p}$
        \item Sea $f(p)=-\sum_{x=1}^{\infty}x(1-p)^x$. Entonces
        $f'(p)=\sum_{x=1}^{\infty}x^2(1-p)^{x-1}$. Por otra parte,
        \begin{longderivation}
            &f(p)\\
            =\\
            &-\sum_{X=1}^{\infty}x(1-p)^x\\
            =\\
            &-(1-p)\sum_{x=1}^{\infty}x(1-p)^{x-1}\\
            =\\
            &p\sum_{x=1}^{\infty}x(1-p)^{x-1}
            -
            \sum_{x=1}^{\infty}x(1-p)^{x-1}\\
            =\\
            &\text{E}[X] - \frac{\text{E}[X]}{p}\\
            =\\
            &\frac{1}{p}-\frac{1}{p^2}
        \end{longderivation}
        Derivando la expresión y multiplicando por $p$, se obtiene
        \begin{longderivation}
            & \text{E}[X^2]\\
            =\\
            &p\odv*{\left(\frac{1}{p}-\frac{1}{p^2}\right)}{p}\\
            =\\
            &\frac{-1}{p}+\frac{2}{p^2}
        \end{longderivation}
        Por la definición de varianza y con el resultado anterior
        se obtiene que 
        \begin{longderivation}
            &\text{Var}[X]\\
            =\\
            &\sum_{x=1}^{\infty}x^2(1-p)^{x-1} - \text{E}^2[X]\\
            =\\
            &\frac{2}{p^2}-\frac{1}{p}-\frac{1}{p^2}\\
            =\\
            &\frac{1-p}{p^2}
        \end{longderivation}
        Con lo que concluye la demostración
    \end{enumerate}
\end{Demo}